\documentclass[10pt]{beamer}
 
\usepackage[T1]{fontenc}
\usepackage[utf8]{inputenc}
\usepackage{graphicx}
\usepackage{hyperref}
\usepackage{lmodern}
\usepackage{listings}
\usepackage{amssymb} 
\usepackage{xcolor}
\usepackage{verbatim}
\usetheme{Warsaw}
\setbeamercovered{transparent}

\author[Latrille Thibault, ENS Lyon]{Latrille Thibault \\Ecole normale supérieure de Lyon, France}
\title[From detection to interpretation of epistasis in GWAS.]{From detection to interpretation of epistasis in Genome Wide Association Studies.}  


% Faire apparaître un sommaire avant chaque section
\AtBeginSection[]{
   \begin{frame}
   \begin{center}{\Large Plan }\end{center}
   %%% affiche en début de chaque section, les noms de sections et
   %%% noms de sous-sections de la section en cours.
   \tableofcontents[currentsection,hideothersubsections]
   \end{frame} 
}
		


\begin{document}

\frame{\titlepage} 

\section{Introduction}


\begin{frame}
\frametitle{Guideline of this talk.}

\begin{center}
\Large
Combining methods from computational biology to detect epistasis and methods from evolutionary biology to interpret epistasis.

\end{center}
\end{frame}

\begin{frame}
\frametitle{Detecting main effects in GWAS.}
	\begin{center}
       \includegraphics[width=11.5cm]{figure11.png}
	\end{center}
\end{frame}

\begin{frame}
\frametitle{Detecting main effects in GWAS.}
	\begin{center}
       \includegraphics[width=11.5cm]{figure12.png}
	\end{center}
\end{frame}

\begin{frame}
\frametitle{Detecting main effects in GWAS.}
	\begin{center}
       \includegraphics[width=11.5cm]{figure13.png}
	\end{center}
\end{frame}

\begin{frame}
\frametitle{Definition of epistasis.}
\Large The combined effect of the factors is higher or less than the addition of the individual main effects.

\end{frame}

\begin{frame}
\frametitle{Formal definition for pairwise epistasis.}

\begin{itemize}
\item<1-> Let us consider two loci with two variant each.
\item<1-> For each locus, one variant is denoted by $0$, the other one by $1$.
\item<1-> The four genotypes possible are :  $00$, $01$, $10$, $11$.
\item<1-> Let us denote their respective related phenotype by $\omega_{00}$, $\omega_{01}$, $\omega_{10}$ and $\omega_{11}$.
\end{itemize}

\begin{center}
\begin{minipage}{7cm}
\begin{description}
\item<2>[Absence of epistasis] $\omega_{00}+\omega_{11}-\omega_{01}-\omega_{10}=0$\\
\item<2>[Positive epistasis] $\omega_{00}+\omega_{11}-\omega_{01}-\omega_{10}>0$\\
\item<2>[Negative epistasis] $\omega_{00}+\omega_{11}-\omega_{01}-\omega_{10}<0$
\end{description}
\end{minipage}
\end{center}


\end{frame}


\section{Detecting epistasis in GWAS, methods from computational biology.}

\begin{frame}
\frametitle{Detecting pairwise epistasis in GWAS.}
	\begin{center}
       \includegraphics[width=10.5cm]{figure2.png}
	\end{center}
\end{frame}

\begin{frame}
\frametitle{Difficulties for detecting epistasis in GWAS}

\begin{itemize}
\Large
\item Multiple testing issues.
\item Computationnaly intensive.
\item Curse of dimensionality.
\item Interpretation is complicated. 
\end{itemize}

\end{frame}

\section{Interpreting epistasis, methods from evolutionary biology.}

\begin{frame}
\frametitle{The shape of the fitness landscape for pairwise epistasis.}
	\begin{center}
       \includegraphics[width=10.5cm]{figure41.png}
	\end{center}
\end{frame}

\begin{frame}
\frametitle{The shape of the fitness landscape for pairwise epistasis.}
	\begin{center}
       \includegraphics[width=10.5cm]{figure42.png}
	\end{center}
\end{frame}

\begin{frame}
\frametitle{The shape of the fitness landscape for pairwise epistasis.}
	\begin{center}
       \includegraphics[width=10.5cm]{figure43.png}
	\end{center}
\end{frame}

\begin{frame}
\frametitle{The shape of the fitness landscape for 3-way epistasis.}
	\begin{center}
       \includegraphics[width=10.5cm]{figure5.png}
	\end{center}
\end{frame}

\section{Discussion.}

\begin{frame}
\frametitle{The complete framework.}
	\begin{center}
       \includegraphics[width=10.5cm]{figure3.png}
	\end{center}
\end{frame}

\begin{frame}

	\begin{center}
	\Huge And if you have been, \\thanks for listening.
	\end{center}
	
\end{frame}

\end{document}


