\documentclass[8pt]{beamer}

\usepackage[T1]{fontenc}
\usepackage[utf8]{inputenc}
\usepackage{hyperref}
\usepackage{listings}
\usepackage{amssymb,amsfonts,amsmath,amsthm}
\usepackage{mathtools}
\usepackage{lmodern}
\usepackage{bm}
\usepackage{xcolor}

%gets rid of bottom navigation overlines
\setbeamertemplate{footline}[frame number]{}

%gets rid of bottom navigation symbols
\setbeamertemplate{navigation symbols}{}

%gets rid of footer
%will override 'frame number' instruction above
%comment out to revert to previous/default definitions
\setbeamertemplate{footline}{}

\sloppy 

\definecolor{RED}{HTML}{EB6231}
\definecolor{YELLOW}{HTML}{E29D26}
\definecolor{BLUE}{HTML}{5D80B4}
\definecolor{LIGHTGREEN}{HTML}{6ABD9B}
\definecolor{GREEN}{HTML}{8FB03E}
\definecolor{PURPLE}{HTML}{BE1E2D}
\definecolor{BROWN}{HTML}{A97C50}
\definecolor{PINK}{HTML}{DA1C5C}

\newcommand{\Data}{{\color{PINK}{D}^{k}}}
\newcommand{\sn}{{\color{BLUE}{S}}}
\newcommand{\s}{{\color{LIGHTGREEN}{S_0^{k}}}}
\newcommand{\si}{{\color{PINK}{S_1^{k}}}}
\newcommand{\sii}{{\color{PINK}{S_2^{k}}}}
\newcommand{\siii}{{\color{PINK}{S_3^{k}}}}
\newcommand{\siiii}{{\color{PINK}{S_4^{k}}}}
\newcommand{\siiiii}{{\color{LIGHTGREEN}{S_5^{k}}}}
\newcommand{\A}{{\color{RED}{\textbf{A}}}}
\newcommand{\D}{{\color{GREEN}{\textbf{D}}}}
\newcommand{\e}{\mathrm{e}}
\newcommand{\ci}{{\color{RED}{\textbf{ATT}}}}
\newcommand{\cj}{{\color{GREEN}\textbf{ATG}}}
\newcommand{\ck}{{\color{BLUE}\textbf{ATA}}}
\newcommand{\nuci}{{\color{RED}\textbf{T}}}
\newcommand{\nucj}{{\color{GREEN}\textbf{G}}}
\newcommand{\nuck}{{\color{BLUE}\textbf{A}}}
\newcommand{\Fi}{{\color{RED}F_{\textbf{Ile}}}}
\newcommand{\Fj}{{\color{GREEN}F_{\textbf{Met}}}}
\newcommand{\aaitoj}{{\color{RED}{\textbf{Ile}}} \rightarrow {\color{GREEN}{\textbf{Met}}}}
\newcommand{\nucitoj}{\nuci \rightarrow \nucj}
\newcommand{\itoj}{\ci \rightarrow \cj}
\newcommand{\nucitok}{\nuci \rightarrow \nuck}
\newcommand{\itok}{\ci \rightarrow \ck}
\newcommand{\der}{\mathrm{d}}
\newcommand{\Ne}{N_\mathrm{e}}
\newcommand{\SetCodon}{\Omega_{\mathrm{C}}}
\newcommand{\SetNuc}{\Omega_{\mathrm{N}}}
\newcommand{\SetWeak}{\Omega_{\mathrm{W}}}
\newcommand{\SetStrong}{\Omega_{\mathrm{S}}}
\newcommand{\SetAa}{\Omega_{\mathrm{A}}}
\newcommand{\Neighbor}{\mathcal{V}}
\newcommand{\NonSyn}{\mathcal{N}}
\newcommand{\Syn}{\mathcal{S}}
\newcommand{\Nx}{\Neighbor_x}
\newcommand{\NxNonSyn}{\NonSyn_x}
\newcommand{\NyNonSyn}{\NonSyn_y}
\newcommand{\NxSyn}{\Syn_x}
\newcommand{\NySyn}{\Syn_y}

% \renewcommand{\baselinestretch}{2}
\usepackage{adjustbox}
\newcommand{\specialcell}[2][c]{%
	\begin{tabular}[#1]{@{}c@{}}#2\end{tabular}}


\begin{document}
	
	\begin{frame}
		\begin{enumerate}
			\item ${q_{\itoj}}$ is the substitution rate from codon $\ci$ to $\cj$.
			\item ${\Ne}$ is the effective population size.
			\item ${\mu_{\nucitoj}}$ is the mutation rate from nucleotide $\nuci$ to $\nucj$
			\item ${p_{\mathrm{fix}}}$ is the probability of fixation.
			\item $\Fi$ ($\Fj$) is the fitness of Isoleucine (Methionine).
		\end{enumerate}
		\vspace{-1em}
		\begin{align*}
			{q_{\itoj}} & { = \Ne \mu_{\nucitoj}  p_{\mathrm{fix}}} \\
			& { = \Ne \mu_{\nucitoj} }  \dfrac{{\Fj - \Fi}}{{1 - \e^{\Ne(\Fi - \Fj)} }}
		\end{align*}
	\end{frame}

	\begin{frame}
		\textbf{\large Wright-Fisher process}
		\begin{enumerate}
			\item Non-overlapping generations, constant population size and random mating.
			\item $S=4 \Ne s$ is the scaled selection coefficient associated to the derived allele.
		\end{enumerate}
		$G(x, S) \der x $ is the expected time for which the population frequency of derived allele is in the range $(x, x+\der x)$ before eventual absorption:
		\begin{align*}
			G(x, S) & = \dfrac{2 \left[ 1 - \e^{-S(1-x)}\right]}{(1 - \e^{-S})x(1-x)}
		\end{align*}
		\textbf{\large Poisson random fields}
		\begin{enumerate}
			\item Each mutation, arising at Poisson times, occurs at a new site (infinite sites)
			\item Each mutation follows an independent (no-linkage) Wright-Fisher process.
			\item $\theta = 4 \Ne \mu $ is the scaled mutation rate of the whole locus.
		\end{enumerate}
		In a sample of size $n$, the expected number of sites with $i$ (which ranges from $1$ to $n-1$) copies of the derived allele is defined as:
		\begin{align*}
	 \dfrac{\theta}{2} \int_{0}^{1} G(x, S) \binom{n}{i} x^{i} (1-x)^{n-i} \der x 	=  \theta \int_{0}^{1} \dfrac{ 1 - \e^{-S(1-x)}}{1 - \e^{-S}} \binom{n}{i} x^{i-1} (1-x)^{n-i-1} \der x
		\end{align*}
	\end{frame}

	\begin{frame}
		\textbf{\large Poisson random fields combined with Mutation-Selection model}
		\begin{enumerate}
			\item $ \bm{\mu} $ is mutation rates matrix between codons (given by the matrix mutations).
			\item $ \bm{F^{(k)}} $ is the vector of scaled amino-acid fitnesses at site $k$.
			\item ${\A}$ is the ancestral codon.
			\item ${\D} \in \mathcal{V}_{{\A}}$ is the derived codon, in the set of neighbors of codon ${\A}$.
			\item $ \theta_{{\A} \rightarrow {\D}} = 4 \Ne \mu_{{\A} \rightarrow {\D}}$ is the scaled mutation rate from ${\A}$ to ${\D}$
			\item $\displaystyle H(i, n, S) = \int_{0}^{1} \dfrac{ 1 - \e^{-S(1-x)}}{1 - \e^{-S}} \binom{n}{i} x^{i-1} (1-x)^{n-i-1} \der x$
		\end{enumerate}
		In a sample of size $n$, the probability of observing $i$ copies of the derived allele (${\D}$), at site $k$, is given by:
		\begin{equation*}
			\begin{dcases}
			& \bullet \ \forall {\D} \in \mathcal{V}_{\A}, \ n \geq i > 0,  \\
			& \hspace{2.5em} P({\A}=n-i,{\D}=i \ |\ \bm{\mu}, \bm{F^{(k)}}, \Ne) = \theta_{{\A} \rightarrow {\D}} H\left(i, n, F_{\D}^{(k)}-F_{\A}^{(k)}\right) \\
			& \bullet \ i=0,  \\
			& \hspace{2.5em}P({\A}=n \ |\ \bm{\mu}, \bm{F^{(k)}}, \Ne) = 1 - \sum_{{\D} \in \mathcal{V}_{\A}} \sum_{i=1}^{n} \theta_{{\A} \rightarrow {\D}} H\left(i, n, F_{\D}^{(k)}-F_{\A}^{(k)}\right) \\
			& 0 \text{  else.}  \\
			\end{dcases}
		\end{equation*}
	\end{frame}

	\begin{frame}
		\begin{enumerate}
			\item $L$ is the number of sites in the coding sequence.
			\item $\pi_{{\A}}^{(k)}$ is equilibrium frequency of ${\A}$ at site $k$.
		\end{enumerate}
		\begin{align*}
		DAF(i, n \ |\ \bm{\mu}, \bm{F}, \Ne) &= \sum_{k=1}^{L} 	\sum_{{\A} \in \mathcal{C}} \pi_{{\A}}^{(k)} \sum_{{\D} \in \mathcal{V}_{\A}} \theta_{{\A} \rightarrow {\D}} H\left(i, n, F_{\D}^{(k)}-F_{\A}^{(k)} \right)
		\end{align*}
	\end{frame}

	\begin{frame}
		\textbf{\large Simulation versus inference }\\
		\vspace{1.5em}
			\begin{center}
				\begin{large}
					\begin{tabular}{|l|l|l|l|l|l|}
					\hline
					$\Ne$ & $\mu$	& $\theta_{\text{simulated}}$ & $\widehat{\theta}_{\text{inferred}}$	& $\textrm{Std} \left(\widehat{\theta}_{\text{inferred}}\right)$ &	Precision (\%)\\
					\hline
					550 & $1.10^{-6}$&	0.0022&	0.00228	&0.000567	& 3.73\\
					\hline
					640 & $1.10^{-6}$&	0.00256&	0.00242&	0.000562&	5.41\\
					\hline
					640 & $1.10^{-6}$&	0.00256&	0.00244&	0.000543&	4.88\\
					\hline
					820 & $1.10^{-6}$&	0.00328&	0.00328&	0.000666&	0.102\\
					\hline
					820 & $1.10^{-6}$&	0.00328&	0.00324&	0.000653&	1.33\\
					\hline
					1000 & $1.10^{-6}$&	0.004&	0.00402	&0.000702	&0.454\\
					\hline
					1000 & $1.10^{-6}$&	0.004&	0.00401&	0.000698	&0.185\\
					\hline
					\end{tabular}
				\end{large}
			\end{center}
		Simulation are using empirically determined fitness vector of Gal4 (64 amino-acids).\\
		$20$ individuals are sampled randomly from each of the $17$ species at the tips of the phylogenetic tree.\\
		\vspace{0.5em}
		Inference has been implemented in a \textit{BayesCode} using interpolations of the function $H(i, n, S)$ to speed-up computations.
	\end{frame}

	\begin{frame}
		\begin{enumerate}
			\item $\tau$ is the tree with given branch lengths ($d_i$).
			\item $\mu$ is mutation matrix ($4\mathrm{x}4$).
			\item $\bm{F}^{k}$ is the fitness vector ($1\mathrm{x}20$) of amino-acids at site $k$.
			\item $\Data=\{\si, \sii, \siii, \siiii \}$ is the observed data at the leaves, at site $k$.
			\item $\s$ and $\siiiii$ are the states of the internal nodes (assumed known), at site $k$.
		\end{enumerate}  
		Likelihood of the data, at site $k$, given the states of the internal nodes is:
		\begin{equation*}
			P\left(\Data| \tau, \mu, \bm{F}^{k}, \s, \siiiii \right) = P_{\s \si}(d_1) \cdot P_{\s \sii}(d_2) \cdot P_{\s \siiiii}(d_5) \cdot P_{\siiiii \siii}(d_3) \cdot P_{\siiiii \siiii}(d_4)
		\end{equation*}
	\end{frame}

	\begin{frame}
		\begin{enumerate}
			\item $\tau$ is the tree with given branch lengths ($d_i$).
			\item $\mu$ is mutation matrix ($4\mathrm{x}4$).
			\item $\bm{F}^{k}$ is the fitness vector ($1\mathrm{x}20$) of amino-acids at site $k$.
			\item $\Data=\{\si, \sii, \siii, \siiii \}$ is the observed data at the leaves, at site $k$.
			\item $\s$ and $\siiiii$ are the unknown states of the internal nodes, at site $k$.
			\item $\pi(\s)$ is the equilibrium frequency of state $\s$.
		\end{enumerate}
		Likelihood of the data at site $k$, given all possible states of internal node is:
		\begin{equation*}
		P\left(\Data| \tau, \mu, \bm{F}^{k}\right) = \sum_{\s=1}^{61}  \sum_{\siiiii=1}^{61} \pi(\s) \cdot P_{\s \si}(d_1) \cdot P_{\s \sii}(d_2) \cdot P_{\s \siiiii}(d_5) \cdot P_{\siiiii \siii}(d_3) \cdot P_{\siiiii \siiii}(d_4) 
		\end{equation*}
	\end{frame}

	\begin{frame}
		For an inner node $i$ with offspring $o_1$ and $o_2$, $L_{i} \left( \sn, \tau, \mu, \bm{F}^{k} \right)$ is defined recursively as:
		\begin{equation*}
		L_{i} \left( \sn, \tau, \mu, \bm{F}^{k} \right) = \left[ \sum_{x=1}^{61} P_{\sn x }(d_{o_1}) L_{o_1}\left( x, \tau, \mu, \bm{F}^{k} \right) \right] \cdot \left[ \sum_{x=1}^{61} P_{\sn x }(d_{o_2}) L_{o_2}\left( x, \tau, \mu, \bm{F}^{k} \right) \right]
		\end{equation*}
		And for a leaf $i$:
		\begin{equation*}
		L_{i}\left( \sn, \tau, \mu, \bm{F}^{k}\right) =
			\begin{dcases}
			1, & \text{if } \sn = {\color{PINK}{S_i^{k}}} \\
			0, & \text{otherwise.}
			\end{dcases}
		\end{equation*}
		Then the likelihood of the data at site $k$, given all possible states of internal node is:
		\begin{equation*}
		P\left(\Data| \tau, \mu, \bm{F}^{k}\right) = \sum_{x=1}^{61} \pi(x) L_{\text{root}} \left( x | \tau, \mu, \bm{F}^{k} \right)
		\end{equation*}
	\end{frame}

	\begin{frame}
		Then the likelihood of the data at site $k$, given all possible states of internal node is:
		\begin{equation*}
		P\left({\color{PINK}{D}^{k}_{1, \cdots, n}}| \tau, \mu, \Ne, \bm{F}^{k}\right) = \sum_{x=1}^{61} \pi(x) L_{\text{root}} \left( x | \tau, \mu, \Ne, \bm{F}^{k} \right)
		\end{equation*}
	\end{frame}
\end{document}
