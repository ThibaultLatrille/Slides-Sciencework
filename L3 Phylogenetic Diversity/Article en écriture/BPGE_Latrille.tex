\documentclass[10pt]{beamer}
 
\usepackage[T1]{fontenc}
\usepackage[utf8]{inputenc}
\usepackage[english]{babel}
\usepackage{graphicx}
\usepackage{hyperref}
\usepackage{lmodern}
\usepackage{listings}
\usepackage{amssymb} 
\usepackage{xcolor}
\usepackage{verbatim}
\usetheme{Warsaw}
\setbeamercovered{transparent}

\date{October 2, 2018}

\author{Latrille Thibault}

\title{Robust estimation of diversity. Steer clear of the pitfall of diversity estimation, rare species.}  
 
\begin{document}

\frame{\titlepage} 

\section{Introduction}

\begin{frame}
\frametitle{An hypothetical (and very simple) example of a sample.}
	\begin{center}
       \includegraphics[width=7.5cm]{infer_sample_1.png}
	\end{center}
\end{frame}

\begin{frame}
\frametitle{And by definition, a sample is only fraction of the community. Sometimes a very tiny one...}
	\begin{center}
       \includegraphics[width=9cm]{infer_sample_2.png}
	\end{center}
\end{frame}

\begin{frame}
\frametitle{Thus, some information are lost forever, and some can still be estimated. Which one?}
	\begin{center}
       \includegraphics[width=9cm]{infer_sample_3.png}
	\end{center}
\end{frame}


\section{How many rare species are they in the community?}

\begin{frame}
\frametitle{How many rare species are they in the community?}
	\begin{center}
       \includegraphics[width=9cm]{infer_species.png}
	\end{center}
	
\end{frame}


\begin{frame}
\frametitle{The Good-Turing estimates, first step.}
	\begin{center}
       \includegraphics[width=8.0cm]{GT_estimates_1.png}
	\end{center}
\end{frame}

\begin{frame}
\frametitle{The Good-Turing estimates, second step.}
	\begin{center}
       \includegraphics[width=8.0cm]{GT_estimates_2.png}
	\end{center}
\end{frame}

\begin{frame}
\begin{center}
Well, we can estimate and evaluate the proportion of unobserved individuals.
But we still do not know in how many species this proportion is divided!
\end{center}
\begin{center}
       \includegraphics[width=8.5cm]{GT_Infer.png}
	\end{center}
\end{frame}

\begin{frame}
\frametitle{Way out : upper and lower estimates.}

	\begin{center}
       \includegraphics[width=8.5cm]{GT_Up_Low.png}
	\end{center}
\end{frame}


\section{Hill diversity}

\begin{frame}
\frametitle{The Hill diversity as a quantitative measure of diversity.}
	\begin{itemize}
		\item<1> $M$ is the number of species in the community.
		\item<1> $x_i$ is the frequency of species $i$ in the community.
		\item<1> $^q D$ is the $q$ hill diversity  of the community.
	\end{itemize}
	\begin{align*}
		^q D = \left( \sum_{i=1}^{M} x_i^q \right)^{1/(1-q)} \\
	\end{align*}
	\begin{itemize}
		\item<1> $^0 D = M$ is the effective number of species of the community.
		\item<1> $^1 D = \mathrm{exp} \left( - \sum_{i=1}^{M} x_i \mathrm{ln} ( x_i )  \right) $ is the Shannon entropy of the community.
		\item<1> $^2 D = 1 / \sum_{i=1}^{M} x_i^2 $ is the Simpson index of the community.
	\end{itemize}
\end{frame}

\begin{frame}
\frametitle{Main results.}
\begin{center}
	\includegraphics[width=9.0cm]{plot.png}
\end{center}
\end{frame}

\begin{frame}
\frametitle{Conclusions.}

	\begin{itemize}
	\item<1> The effective number of species ($^0 D$) is unreliable.
	\item<1> The Simpson index ($^2 D$) is very robust since unobserved but rare species do not affect it.
	\item<1> The Shannon entropy ($^1 D$) is exactly where the sample size have a determinant effect on the robustness of the estimator.
	\end{itemize}
	
\end{frame}

\end{document}


