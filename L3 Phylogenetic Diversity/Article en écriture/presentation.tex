\documentclass[10pt]{beamer}
 
\usepackage[T1]{fontenc}
\usepackage[utf8]{inputenc}
\usepackage[french]{babel}
\usepackage{graphicx}
\usepackage{hyperref}
\usepackage{lmodern}
\usepackage{listings}
\usepackage{amssymb} 
\usepackage{xcolor}
\usepackage{verbatim}
\usetheme{Warsaw}
\setbeamercovered{transparent}
\author{Latrille Thibault}


\title{Robust estimation of phylogenetic diversity. Steer clear of the pitfall of diversity estimation, rare species.}  
 
\begin{document}

\frame{\titlepage} 

\section{Introduction}

\begin{frame}
\frametitle{An hypothetical (and very simple) example of a sample.}
	\begin{center}
       \includegraphics[width=7.5cm]{infer_sample_1.png}
	\end{center}
\end{frame}

\begin{frame}
\frametitle{And by definition, a sample is only fraction of the community. Sometimes a very tiny one...}
	\begin{center}
       \includegraphics[width=9cm]{infer_sample_2.png}
	\end{center}
\end{frame}

\begin{frame}
\frametitle{Thus, some information are lost forever, and some can still be estimated. Which one?}
	\begin{center}
       \includegraphics[width=9cm]{infer_sample_3.png}
	\end{center}
\end{frame}


\section{How many rare species are they in the community?}

\begin{frame}

	\begin{center}
       \includegraphics[width=9.5cm]{infer_species.png}
	\end{center}
	
\end{frame}


\begin{frame}
\frametitle{The Good-Turing estimates, first step.}
	\begin{center}
       \includegraphics[width=8.0cm]{GT_estimates_1.png}
	\end{center}
\end{frame}

\begin{frame}
\frametitle{The Good-Turing estimates, second step.}
	\begin{center}
       \includegraphics[width=8.0cm]{GT_estimates_2.png}
	\end{center}
\end{frame}

\begin{frame}
\begin{center}
Well, we can estimate and evaluate the proportion of unobserved individuals.
But we still do not know in how many species this proportion is divided!
\end{center}
\begin{center}
       \includegraphics[width=8.5cm]{GT_Infer.png}
	\end{center}
\end{frame}

\begin{frame}
\frametitle{Way out : upper and lower estimates}

	\begin{center}
       \includegraphics[width=8.5cm]{GT_Up_Low.png}
	\end{center}
\end{frame}

\section{Workflow}
\frametitle{Estimation of phylogenetic abundance trees.}
\begin{frame}

	\begin{center}
       \includegraphics[width=11.5cm]{flow.png}
	\end{center}

\end{frame}


\section{Phylogenetic Hill diversity}
\frametitle{The Hill diversity in less than 30 seconds.}

\begin{frame}
	\begin{center}
       \includegraphics[width=11.0cm]{Tree.png}\\

	\end{center}
\end{frame}



\begin{frame}
\frametitle{Main results.}
	\begin{center}
       \includegraphics[width=9.0cm]{plot.png}
	\end{center}
\end{frame}



\begin{frame}[cc]
\frametitle{Conclusion}

	\begin{itemize}
	\item<1> The \textbf{phylogenetic} diversity for $q=0$ is unreliable (PD by Faith).
	\item<1> The \textbf{phylogenetic} Rao's quadratic entropy ($q=2$) is very robust since unobserved but rare species do not affect it.
	\item<1> The \textbf{phylogenetic} Shannon's index is exactly where the sample size have a determinant effect on the robustness of the estimator.
	\end{itemize}
	
\end{frame}

\end{document}


