\documentclass[10pt]{beamer}
 
\usepackage[T1]{fontenc}
\usepackage[utf8]{inputenc}
\usepackage[french]{babel}
\usepackage{graphicx}
\usepackage{hyperref}
\usepackage{lmodern}
\usepackage{listings}
\usepackage{amssymb} 
\usepackage{xcolor}
\usepackage{verbatim}
\usetheme{Warsaw}
\setbeamercovered{transparent}


\author{Thibault Latrille, Jérôme Hamelin, Bart Haegeman}
\title{Robust estimation of phylogenetic diversity}  
\institute{Laboratoire de Biotechnologie de l'Environnement (LBE), INRA Narbonne}
\date{9 August 2013}
 
\begin{document}

\frame{\titlepage} 

\section{Introduction}

\frame{\tableofcontents}

\begin{frame}
\frametitle{An hypothetical (and very simple) example of a sample.}
	\begin{center}
       \includegraphics[width=7.5cm]{infer_sample_1.png}
	\end{center}
\end{frame}

\begin{frame}
\frametitle{And by definition, a sample is only fraction of the community. Sometimes a very tiny one...}
	\begin{center}
       \includegraphics[width=9cm]{infer_sample_2.png}
	\end{center}
\end{frame}

\begin{frame}
\frametitle{Thus, some information are lost forever, and some can still be estimated. but which one ?}
	\begin{center}
       \includegraphics[width=9cm]{infer_sample_3.png}
	\end{center}
\end{frame}


\section{A case study : Species Richness}

\begin{frame}
\frametitle{A case study : Species Richness}
The Species Richness is simply the number of different species in the sample.

	\begin{center}
       \includegraphics[width=5.5cm]{infer_species.png}
	\end{center}
	
\begin{itemize}
\item Use the number of species in the sample as an estimate for the community... Clearly under estimate the real value.
\item Fit a distribution (Log-Normal, Power-law,...) to the sample and extrapolate to the size of the community. 
\end{itemize}
\end{frame}


\begin{frame}
\frametitle{The Good-Turing estimates, first step.}
	\begin{center}
       \includegraphics[width=8.0cm]{GT_estimates_1.png}
	\end{center}
\end{frame}

\begin{frame}
\frametitle{The Good-Turing estimates, second step.}
	\begin{center}
       \includegraphics[width=8.0cm]{GT_estimates_2.png}
	\end{center}
\end{frame}

\begin{frame}
\begin{center}
Well, we can estimate and evaluate the proportion of unobserved individuals.
But we still do not know in how many species this proportion is divided !
\end{center}
\begin{center}
       \includegraphics[width=8.5cm]{GT_Infer.png}
	\end{center}
\end{frame}

\begin{frame}
\frametitle{Way out : Upper and Lower estimates}

	\begin{center}
       \includegraphics[width=8.5cm]{GT_Up_Low.png}
	\end{center}
\end{frame}

\section{Hill diversity}

\begin{frame}
\frametitle{Generalize diversity measures : the Hill diversity}
\begin{center}
 $ \displaystyle D^{\alpha} =\left( \sum_{i=1}^{S} a_i^{\alpha} \right)^{1/(1-\alpha)}$
\end{center}
Where $a_i$ is the relative abundance (fraction) of species $i$ $(i\in \{1, ... S\})$.
	
\begin{itemize}
\item 	input : a vector of relative abundance and a positive real number for the parameter $\alpha$.
\item output : a positive real value.
\end{itemize}

\begin{itemize}
\item $D^0 \rightarrow S $ 
\item $D^1 \rightarrow $ Shannon's index 
\item $D^2 \rightarrow $ Simpson's index
\end{itemize}

\end{frame}

\begin{frame}

	\begin{center}
	$ \displaystyle  D^{\alpha} 
 =\left( \sum_{i=1}^{S} a_i^{\alpha} \right)^{1/(1-\alpha)}$
       \includegraphics[width=9cm]{plot.png}
	\end{center}
\end{frame}

\begin{frame}
	\begin{center}
	$ \displaystyle  D^{\alpha} 
 =\left( \sum_{i=1}^{S} a_i^{\alpha} \right)^{1/(1-\alpha)}$
       \includegraphics[width=9cm]{plot_UL.png}
	\end{center}
\end{frame}


\begin{frame}[cc]
\uncover<1>{\textbf{Intermediate Conclusion} \footnote[1]{J.Hamelin \& B. Haegeman, Robust estimation of microbial diversity in theory and in practice} :}
	\begin{itemize}
	\item<1> The Species Richness, although widely used, is unreliable...
	\item<1> The Simpson's index is very robust since unobserved but rare species do not affect it.
	\item<1> The Shannon's index is exactly where the upper estimate takes off and soars.
	\end{itemize}
	
	\uncover<2>{Let us add a new complexity and dimensionality to the problem, the underlying phylogenetic tree...}
	
	
\end{frame}



\section{Phylogenetic Hill diversity}
\frametitle{An hypothetical and still very simple phylogenetic tree}
\begin{frame}

	\begin{center}
       \includegraphics[width=11.0cm]{Tree_1.png}
	\end{center}

\end{frame}

\begin{frame}
	\begin{center}
	 $\displaystyle \bar{T}= \sum_{i \in \mathcal{B}} \mathbin{\textcolor{red}{L_i}} \mathbin{\textcolor{blue}{a_i}}$\\
       \includegraphics[width=7.0cm]{Tree.png}\\
       Where $\mathbin{\textcolor{red}{L_i}}$ is the length of branch from node i to his parent, $ \mathbin{\textcolor{blue}{a_i}}$ is the relative abundance of node i $( i \in \mathcal{B}$, the set of all nodes$)$.
	\end{center}
\end{frame}

\begin{frame}
	\begin{center}
	 $ \displaystyle \bar{D}^{\alpha} =\left( \sum_{i \in \mathcal{B}} \dfrac{ \mathbin{\textcolor{red}{L_i}} \mathbin{\textcolor{blue}{a_i}}^{\alpha}}{\bar{T}} \right)^{1/(1-\alpha)} $, where $\displaystyle \bar{T}= \sum_{i \in \mathcal{B}} \mathbin{\textcolor{red}{L_i}} \mathbin{\textcolor{blue}{a_i}}$\\
       \includegraphics[width=7.0cm]{Tree.png}\\
       Where $\mathbin{\textcolor{red}{L_i}}$ is the length of branch from node i to his parent, $ \mathbin{\textcolor{blue}{a_i}}$ is the relative abundance of node i $( i \in \mathcal{B}$, the set of all nodes$)$.
	\end{center}
\end{frame}




\begin{frame}
The Good-Turing lower and upper estimates tell us how many unobserved species we have, but it does not tell us anything about the \textbf{branching} of these new species.
\begin{itemize}
\item Get extreme estimates : \\
	We do not conserve the structure and topology of the tree.
\item Add the leaves randomly but parsimoniously on the tree : \\
	We still conserve the structure and topology of the tree.
\end{itemize}

\end{frame}


\begin{frame}
	\begin{center}
       \includegraphics[width=10.0cm]{Workflow_1.png}\\
              The number of unobserved tips (species) in the reconstructed community is always given by the upper Good-Turing estimates.
	\end{center}
\end{frame}

\begin{frame}
	\begin{center}
       \includegraphics[width=10.0cm]{Workflow_2.png}\\
       The number of unobserved tips (species) in the reconstructed community is always given by the upper Good-Turing estimates.
	\end{center}
\end{frame}

\begin{frame}
	\begin{center}
       \includegraphics[width=10.0cm]{Workflow_3.png}\\
              The number of unobserved tips (species) in the reconstructed community is always given by the upper Good-Turing estimates.
	\end{center}
\end{frame}

\begin{frame}
	\begin{center}
       \includegraphics[width=10.0cm]{Workflow_4.png}\\
              The number of unobserved tips (species) in the reconstructed community is always given by the upper Good-Turing estimates.
	\end{center}
\end{frame}


\begin{frame}
\frametitle{Test $H_0$ : the trees are sampled from the same community.}
\begin{itemize}
\item FST-test : \\
	Test if the trees display signs of clusters.
\item P-test : \\
	Test if the nodes are randomly assigned.
\item UniFrac-test : \\
	Test if the UniFrac distance is as far as the distance between to samples.
\item Diversity-test : \\
	Test if the trees have the same diversity.
\end{itemize}

\end{frame}


\begin{frame}

\frametitle{An example : Diversity test.}
	\begin{center}

       \includegraphics[width=10.0cm]{diversity_test.png}
	\end{center}
\end{frame}

\begin{frame}

\frametitle{Results.}
Outputs of tests are : \\

	\begin{itemize}
	\item<1> Maximum and minimum estimates do not conserve basics attributes of the tree such as topology, clustering, diversity.
	\item<1> On the other hand, parsimonious upper and lower estimates do conserve these attributes.
	\end{itemize}

\end{frame}

\begin{frame}

\frametitle{An example : Diversity test.}
	\begin{center}

       \includegraphics[width=11.0cm]{diversity_res.png}
	\end{center}
\end{frame}


\begin{frame}

\frametitle{Results.}
The robustness of estimates are : \\

	\begin{itemize}
	\item<1> The diversity for $\alpha=2$ is well estimated by the minimum and maximum estimates.
	\item<1> The range between the parsimonious upper and lower estimates for a diversity of $\alpha=0$ is very large.
	\item<1> The range between the parsimonious upper and lower estimates for a diversity of $\alpha=1$ is not large but the one for the minimum and maximum estimates is.
	\end{itemize}

\end{frame}


\begin{frame}[cc]
\frametitle{Conclusion}

	\begin{itemize}
	\item<1> The \textbf{Phylogenetic} Diversity for $\alpha=0$ is unreliable.
	\item<1> The \textbf{Phylogenetic} Simpson's index is very robust since unobserved but rare species do not affect it.
	\item<1> The \textbf{Phylogenetic} Shannon's index is exactly where the topology of the tree can have a determinant effect on the robustness of the estimator.
	\end{itemize}
	
One need to add randomness to the process reconstruction to get a liable estimate.
	
	
\end{frame}


\end{document}


