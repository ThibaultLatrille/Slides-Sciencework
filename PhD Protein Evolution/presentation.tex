\documentclass[8pt]{beamer}

\usepackage[T1]{fontenc}
\usepackage[utf8]{inputenc}
\usepackage{graphicx}
\usepackage{hyperref}
\usepackage{lmodern}
\usepackage{listings}
\usepackage{amssymb} 
\usepackage{xcolor}
\usepackage{tikz}
\usepackage{bm}
\usebackgroundtemplate{\tikz\node[opacity=0]{};}
\usepackage{adjustbox}

%gets rid of bottom navigation overlines
\setbeamertemplate{footline}[frame number]{}

%gets rid of bottom navigation symbols
\setbeamertemplate{navigation symbols}{}

%gets rid of footer
%will override 'frame number' instruction above
%comment out to revert to previous/default definitions
\setbeamertemplate{footline}{}

\sloppy 

\begin{document}
	
	\begin{frame}
		\begin{enumerate}
			\item $ f(\Delta G_i^{(k,t)}) $  is the fitness of the sequence $\mathbb{S}^t$ at time $t$, given amino-acid $i$ at site $k$. \\
			\item $ \pi_i^{(k,t)} = \dfrac{ e^{2 N_\mathrm{e} f(\Delta G_i^{(k,t)})} }{ \sum_{j=1}^{20} e^{2 N_\mathrm{e} f(\Delta G_j^{(k,t)})} }$ is the propensity of amino-acid $i$ at site $k$ and time $t$\\Thus $ \pi_i^{(k,t)} $ is the probability of observing $i$ at equilibrium ($0 \leq \pi_i^{(k,t)} \leq 1$). \\
		\end{enumerate}
		$ \pi_i^{(168, t)}$
	\end{frame}

	\begin{frame}
		\begin{table}[ht]
			\centering
			\begin{adjustbox}{width = 1\textwidth}
				\begin{tabular}{|c|c|c|c|}
					\hline
				 & Buried site & Partially buried site & Exposed site \\
					\hline
				$ \langle \Omega( \boldsymbol{\pi})^{(k,t)} \rangle $ & $2.94$ & $8.44$ & $10.48$ \\
				\hline
				$ \Omega (\langle \boldsymbol{\pi}^{(k,t)} \rangle) $ & $6.83$ & $14.52$ & $18.71$ \\
					\hline
				\end{tabular}
			\end{adjustbox}
			\caption{We can quantify the degree to which a site is constrained by considering the effective effective number of amino-acids ($\Omega$). $ \langle \Omega( \boldsymbol{\pi}^{(k,t)}) \rangle $ is the average effective number of amino-acids over the simulation. $ \Omega (\langle \boldsymbol{\pi}^{(k,t)} \rangle) $ is the effective number of amino-acids given the averaged propensities over the simulation. }
		\end{table}
		\begin{enumerate}
			\item $ \boldsymbol{\pi}^{(k,t)} $ is the vector of  amino-acid propensities at site $k$ and time $t$. \\
			\item $ \Omega(\boldsymbol{\pi}^{(k,t)}) = e^{ - \sum_{i=1}^{20} \pi_i^{(k,t)} ln(\pi_i^{(k,t)}) } $ is the effective number of amino-acids at site $k$ and time $t$ ($1 \leq \Omega \leq 20$).
		\end{enumerate}
	
	\end{frame}
	
	\begin{frame}
		$\beta$ is the inverse of the temperature ($\beta=1/T$)  \\
		$N_\mathrm{e}$ is effective population size  \\
		$\mu$ is the mutation rate  \\
		$\mathbb{S}^t$ is the nucleotide sequence at time $t$ \\
		$G_{\mathrm{F}}(\mathbb{S}^t)$ is the free energy of the folded state  \\
		$G_{\mathrm{U}}(\mathbb{S}^t)$ is the free energy of the unfolded state  \\
		$\Delta G^t = G_{\mathrm{F}}(\mathbb{S}^t) - G_{\mathrm{U}}(\mathbb{S}^t)$ \\
		$ f = \dfrac{P_{\mathrm{F}}}{P_{\mathrm{F}} + P_{\mathrm{U}}} = \dfrac{e^{-\beta G_{\mathrm{F}}}}{e^{-\beta G_{\mathrm{F}}} + e^{-\beta G_{\mathrm{U}}}} = \dfrac{1}{1 + e^{\beta \Delta G}} $ \\
		$ f (\Delta G^t) =  \dfrac{1}{1 + e^{\beta \Delta G^t}} $ is the fitness function\\
		$\mathbb{S}^{t+1}$ is the nucleotide sequence at time $t+1$ \\
		$\Delta G^{t+1}$ \\
		$ f(\Delta G^{t+1})$ \\
		$ s =  \dfrac{f(\Delta G^{t+1}) - f(\Delta G^{t})}{f(\Delta G^{t})} $ is the selection coefficient of the mutant\\
		$ p_{\mathrm{fixation}} =   \dfrac{1 - e^{-2s} }{1 - e^{-4 N_\mathrm{e} s}} $ is the probability of fixation of the mutant\\
		$ q = \mu * p_{\mathrm{fixation}} $ is the substitution rate\\
	\end{frame}
	
\end{document}


