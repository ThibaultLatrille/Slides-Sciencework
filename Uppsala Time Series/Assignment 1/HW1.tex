% Preview source code

%% LyX 2.0.5 created this file.  For more info, see http://www.lyx.org/.
%% Do not edit unless you really know what you are doing.
\documentclass{article}
\usepackage{lmodern}
\renewcommand{\ttdefault}{lmodern}
\usepackage[T1]{fontenc}
\usepackage[utf8]{inputenc}
\usepackage{float}
\usepackage{amsmath}
\usepackage{amssymb}
\usepackage{graphicx}

\makeatletter

%%%%%%%%%%%%%%%%%%%%%%%%%%%%%% LyX specific LaTeX commands.
%% A simple dot to overcome graphicx limitations
\newcommand{\lyxdot}{.}


%%%%%%%%%%%%%%%%%%%%%%%%%%%%%% User specified LaTeX commands.


%\usepackage[french]{babel}
\usepackage{amsthm}\usepackage{listings}\usepackage{stmaryrd}\usepackage{enumerate}\usepackage{dsfont}\usepackage{cancel}\usepackage{amsfonts}\usepackage{fullpage}

\author{Thibault Latrille \and Yiming Zhang}
\title{Time series
 \\ Home assignment 1}  




\makeatother

\begin{document}
\maketitle

\paragraph{Exercise 1.}
\
\subparagraph{a)}
Our time series is from \textbf{Wolfram|Alpha\textsuperscript{TM}} (www.wolframalpha.com), an online computational knowledge engine developed by Wolfram Research and based on \textbf{mathematica\textsuperscript{TM}}. The engine is capable of responding to factual queries directly by computing the answer from structured data, rather than providing a list of documents or web pages that contain the answer as a search engine might.\\

 The "answer engine" query on a structured database stored on the Wolfram|Alpha servers but can also gather data from other website as well such as wikipedia, CIA's World Factbook, United States Geological Survey, Facebook, etc. 
\\

The query we submitted to the engine was "Evolution of the name anthony". Among many information, the engine returned a time series from 1880 to 2011 of the yearly number of babies named Anthony born in United State each year, normalized by the total number of babies born this year. 
\begin{figure}[H]
	\centering
  	\includegraphics[width=1\textwidth]{anthony.png}
  	\caption{\textbf{Snapshot of the time series inside mathematica.} }
\end{figure}

The data have been exported by \textbf{mathematica\textsuperscript{TM}} and then formatted by a \textbf{Python} script in order to be imported in \textbf{R} with the function \textit{read.table()}. Our analysis can be easily adapted to suit the same investigation with other names, as well as using the number of birth instead of the fraction. The engine can gather and format a vast array of data, and thus \textbf{Wolfram|Alpha\textsuperscript{TM}} represent a tool of partiality for time series.
\subparagraph{b)}
\


\begin{figure}[H]
	\centering
  	\includegraphics[width=0.60\textwidth]{anthony_0.png}
  	\caption{\textbf{Time series plot of the data imported in R along with the trend estimation (red).} }
\end{figure}

Since there is no simple functions (linear, polynomial, exponential, etc) that seems to fit the trend, we rather prefer to use a moving average (Spencer filter) to estimate the trend.
\begin{footnotesize}
\begin{verbatim}
###### R-code #########
## Imporation and plot of the data
anthony=ts(read.table('C:/Users/Thibault/Desktop/Time Series/anthony.txt'))
year<-seq(1880,2011,1)
plot(year,anthony,col=1,lwd=2)
##################
## Moving average with a Spencer filter and plot of the trend
f<-c(-3,-6,-5,3,21,46,67,74,67,46,21,3,-5,-6,-3)/320
trend<-filter(anthony,f)
lines(year,trend,col=2,lwd=2)
\end{verbatim} 
\end{footnotesize}
\subparagraph{c)}
\
\begin{figure}[H]
	\centering
  	\includegraphics[width=0.60\textwidth]{anthony_1.png}
  	\caption{\textbf{Time series plot of the residuals.} The detrended time series is obtained by subtracting the trend to the original data. }
\end{figure}
\begin{footnotesize}
\begin{verbatim}
###### R-code #########
## Compute detrended time series and plot
res=ts(anthony[8:(n-7)]-trend[8:(n-7)])
plot(res)
\end{verbatim} 
\end{footnotesize}

\subparagraph{d)}
\
\begin{figure}[H]
	\centering
  	\includegraphics[width=0.55\textwidth]{anthony_2.png}
  	\caption{\textbf{Empirical auto-correlation function of the detrended time series.} Our empirical auto-correlation function does not seem to fit the acf of an auto-regressive process (exponential decay of the acf) neither a moving average process (peak for n=0 and n=1 only). Moreover, the empirical acf is close to $0$ for $n \geq 1$, thus we can interpret it as a white noise. The trend contains sufficient information about the raw data, the residuals is interpreted as a white noise and does not contain information. }
\end{figure}
\begin{footnotesize}
\begin{verbatim}
###### R-code #########
## empirical auto-correlation function of the residuals
acf(res)
\end{verbatim} 
\end{footnotesize}

\paragraph*{Exercise 2}
\

\subparagraph*{a)}

Our data shows the number of suicides each month that occurred in
Taiwan between 1991 and 2011. The data came from the Department of
Health, Republic of China (Taiwan) Health Statistics Database (http://www.doh.gov.tw).
Data was compiled from annual reports in the database and imported
into R using the $read.ts()$ function. A plot of the data was also
made in R using the $plot()$ function, as shown in figure 5. Visually,
one can see that there is an upward trend in the data and a clear
seasonal pattern.

\begin{figure}[H]
\centering
\includegraphics[scale=0.5]{\string"Suicide Data Plot\string".pdf}

\caption{Time Series Plot of Monthly Number of Suicides in Taiwan (1991-2011)}


\end{figure}


\texttt{\#\#\#\#\#\# R-code \#\#\#\#\#\#\#\#\#}

\texttt{\#\# Reading Data }

\texttt{suicide <- read.ts(\textquotedbl{}/Users/yimingzhang0107/Documents/Course/Uppsala
University/Analysis of Time Series/Suicide Data.txt\textquotedbl{}) }

\texttt{suicide <- ts(suicide,frequency=12) }

\texttt{\#\# Plotting Time Series }

\texttt{plot(suicide,col=1,lwd=2)}


\subparagraph*{b)}

We apply the classical decomposition on the time series using the
$stl()$ function in R. Results are shown below. As shown in the plot
(figure 6), the time series is decomposed into seasonal components,
trend components and the remainder deseasonalized detrended components.

\begin{figure}[H]
\centering
\includegraphics[scale=0.5]{\string"Decomp\string".pdf}
\caption{Classical Decomposition of Time Series}
\end{figure}


\texttt{\#\#\#\#\#\# R-code \#\#\#\#\#\#\#\#\#}

\texttt{\#\# Decomposition }

\texttt{suicideDecomp <- stl(suicide, \textquotedbl{}periodic\textquotedbl{}) }

\texttt{plot(suicideDecomp)}


\subparagraph{c)}

We interpret from our data that the seaonal variation within our data
is related to the natural seasons (Spring, Summer, Autumn, Winter).
We further attempt to apply differencing on the original time series
to detrend and deseasonalize the series. Results are shown below in
figure 7. It can be seen that the data appears detrended and deseasonalized

\begin{figure}[H]
\centering\includegraphics[scale=0.5]{\string"Diff Plot\string".pdf}

\caption{Differencing Time Series}


\end{figure}


\texttt{\#\#\#\#\#\# R-code \#\#\#\#\#\#\#\#\#}

\texttt{\#\# Differencing}

\texttt{suicideDiff <- diff(suicide, lags=3) }

\texttt{plot(suicideDiff)}


\subparagraph{d)}

Comparing the two deseasonalized series. We see that both methods
managed to eliminate seasonal and trend components. Whereas the decomposition
method estimated and subtracted the trend and seasonal components
from our series. The differencing method removed these components
from our data by repeatedly differencing at one or more lags to generate
a noise sequence.


\subparagraph{e)}

Using the $acf()$ function in R. We plot the autocorrelation function
of the deseasonalized series as shown below.

\texttt{\#\#\#\#\#\# R-code \#\#\#\#\#\#\#\#\#}

\texttt{\#\# Plot ACF}

\texttt{suicideACF <- acf(suicideDiff)}

\begin{figure}[H]
\centering\includegraphics[scale=0.5]{\string"DiffACF Plot\string".pdf}

\caption{ACF of Deseaonalized Detrended Time Series}


\end{figure}


Here, the autocorrelation function does not resemble an autoregressive
process (exponential decay) nor a moving average process (peaks at
$n=0$ and $n=1$ only). But for $n\geq1$, the acf is close to zero
and thus can be interpreted as white noise.
\end{document}