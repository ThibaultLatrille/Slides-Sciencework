\documentclass{article}

\usepackage[T1]{fontenc}
\usepackage[utf8]{inputenc}
%\usepackage[french]{babel}
\usepackage{graphicx}
\usepackage{lmodern}
\usepackage{amsmath}
\usepackage{amsthm}
\usepackage{listings}
\usepackage{stmaryrd}
\usepackage{enumerate}
\usepackage{amssymb}
\usepackage{dsfont}
\usepackage{cancel}
\usepackage{amsfonts}
\usepackage{float}
\usepackage{fullpage}

\author{Thibault Latrille \and Yiming Zhang}
\title{Time series
 \\ Home assignment 1}  


\begin{document}
\maketitle

\paragraph{Exercise 1.}
\

Our data is from the \textbf{Mauna Loa Observatory} (MLO), an atmospheric baseline station on Mauna Loa volcano, on the big island of Hawaii. 
This observatory has been harvesting and monitoring data relating to atmospheric change since 1956. \\

Especially, this center has provided the oldest and most accurate data we own related to atmospheric carbon dioxide ($CO_2$). Indeed, because located far from any continent, the air sampled is a good average for the central pacific. Being high, it is above the inversion layer where most of the local effects are present (Wikipedia). 
\\


The data set can be found at \textit{http://cdiac.ornl.gov/trends/co2/sio-keel.html} along with other date relating to atmospheric change. It is a yearly average of $CO_2$ concentration (ppmv) from 1959 to 2009. This data set shows  no seasonal trend since it is a yearly average. \\

However, at a geological time scale, $CO_2$ concentration shows a seasonal trend, but our time span of 50 years is too short to even have a glimpse on it. Using ice core samples in the Antarctic or Greenland ice sheets and bubbles of air (fluid or gas inclusions) trapped in it provide reliable data on atmospheric composition eons ago.

\begin{figure}[H]
	  \centering
  	\includegraphics[width=0.60\textwidth]{MLO.png}
  	\caption{\textbf{Time series plot.} }
	\end{figure}


\end{document}











