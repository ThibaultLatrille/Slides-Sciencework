\documentclass{article}

\usepackage[T1]{fontenc}
\usepackage[utf8]{inputenc}
%\usepackage[french]{babel}
\usepackage{graphicx}
\usepackage{lmodern}
\usepackage{amsmath}
\usepackage{amsthm}
\usepackage{listings}
\usepackage{stmaryrd}
\usepackage{enumerate}
\usepackage{amssymb}
\usepackage{cancel}
\usepackage{amsfonts}
\usepackage{float}
\usepackage{fullpage}

\author{Latrille Thibault\\
\small thibault.latrille@ens-lyon.fr\\[-0.8ex]
\small Uppsala Universitet\\}
\title{Regression and analysis of variance \\ Home assignment 4}  
\newtheorem*{theo}{Theorem} 
\begin{document}
\maketitle
\paragraph{Exercise 1.}
\subparagraph{a)}

The response surface design is the \textit{hexagon design}, we incorporate curvature using the second-order model :
\begin{center}
$y=\beta_0+ \beta_{1} x_{1} + \beta_{2} x_{2}+ \beta_{11} x_{1}^2+ \beta_{22} x_{2}^2 + \beta_{12} x_{1} x_{2} +\epsilon$, where $\epsilon \sim \mathcal{N}(0,\sigma^2)$ represent the noise.
\end{center}

The replicates at the center are used to estimate $\sigma$. The following table contains the estimates of the coefficients and the ANOVA table for the experiment.
\begin{small}
\begin{verbatim}
rsm(formula = y1 ~ SO(x1, x2), data = T)

Residuals:
     1      2      3      4      5      6      7      8      9     10     11 
-1.333  1.333 -1.333  1.333 -1.333  1.333 -1.800  0.200 -2.800 -4.800  9.200 

Coefficients:
            Estimate Std. Error t value Pr(>|t|)    
(Intercept)   59.800      2.276  26.278 1.49e-06 ***
x1             5.333      2.938   1.815   0.1292    
x2             1.732      2.938   0.590   0.5811    
x1:x2          1.155      5.876   0.197   0.8519    
x1^2           4.200      4.257   0.987   0.3692    
x2^2           9.533      4.257   2.239   0.0753 .  
---
Signif. codes:  0 ‘***’ 0.001 ‘**’ 0.01 ‘*’ 0.05 ‘.’ 0.1 ‘ ’ 1 

Residual standard error: 5.089 on 5 degrees of freedom
Multiple R-squared: 0.6545,     Adjusted R-squared: 0.309 
F-statistic: 1.894 on 5 and 5 DF,  p-value: 0.25 

Analysis of Variance Table

Response: y1
            Df  Sum Sq Mean Sq F value Pr(>F)
FO(x1, x2)   2  94.333  47.167  1.8216 0.2545
TWI(x1, x2)  1   1.000   1.000  0.0386 0.8519
PQ(x1, x2)   2 149.927  74.964  2.8951 0.1462
Residuals    5 129.467  25.893               
Lack of fit  1  10.667  10.667  0.3591 0.5813
Pure error   4 118.800  29.700
\end{verbatim}
\end{small} 
All parameters are non significantly different from 0 at the level test of significance $\alpha=0.05$, except $\widehat{\beta_0}$ with p-value equal 1.49e-06.
\begin{figure}[H]
	  \centering
  	\includegraphics[width=0.8\textwidth]{4_1.png}
  	\caption{\textbf{Response surface and contour plot fitted to the data} \textsl{The red dot represent the minimum of the response surface}
  	}
	\end{figure} 
\subparagraph{b)}
Writting the model in matrix notation, we have
\begin{center}
$\widehat{y}=\widehat{\beta_0}+\pmb{x}^T\pmb{b} +\pmb{x}^T\pmb{B}\pmb{x}$
\end{center}
Where
\begin{align*}
\pmb{X}=
\begin{pmatrix}
x_1 \\
x_2
\end{pmatrix}
\qquad
\pmb{b}=
\begin{pmatrix}
\widehat{\beta_1} \\
\widehat{\beta_2} \\
\end{pmatrix}
\qquad
\pmb{B}=
\begin{pmatrix}
\widehat{\beta_{11}} & \dfrac{\widehat{\beta_{12}}}{2} \\
\dfrac{\widehat{\beta_{12}}}{2} & \widehat{\beta_{22}} \\
\end{pmatrix}
\end{align*}
The stationary point $\pmb{x}_s$ satisfy the equation :
\begin{align*}
\left( \dfrac{\partial\widehat{y}}{\partial\pmb{x}}\right)_{\pmb{x}_s}=\pmb{b}+2\pmb{B}\pmb{x}=0
\iff \pmb{x}_s=-\dfrac{1}{2}\pmb{B}^{-1}\pmb{b}=\begin{pmatrix}
-0.628 \\
-0.053 \\
\end{pmatrix}
\end{align*}
Furthermore, 
\begin{center}
$\widehat{y_s}=\widehat{\beta_0}+\dfrac{1}{2}\pmb{x}_s^T\pmb{b}=58.08$
\end{center}



\paragraph{Exercise 2.}
\subparagraph{a)}
The model for the the data is \begin{center}
$$
y_{ijk}=\mu +\tau_i+\beta_j+ \gamma_k + (\tau\beta)_{ij}+ (\tau\gamma)_{ik}+ (\beta\gamma)_{jk}+ (\tau\beta\gamma)_{ijk} +\epsilon_{ijk} \left \{
    \begin{array}{ll}
        i=1,2 \\
		j=1,2,3 \\
		k=1,2,3 
    \end{array}
\right.
\text{ with } \epsilon_{ijk} \sim \mathcal{N}(0,\sigma^2)
$$
\end{center}
\begin{center}
$$\sum_i\tau_i=\sum_j\beta_i=\sum_k\gamma_k=0,$$
\end{center}
\begin{center}
 $$ \sum_i (\tau\beta)_{ij} = \sum_j (\tau\beta)_{ij} = \sum_i(\tau\gamma)_{ik}=\sum_k (\tau\gamma)_{ik}=\sum_j(\beta\gamma)_{jk}=\sum_k (\beta\gamma)_{jk}=0,$$
\end{center}
 \begin{center}
$$ \sum_i (\tau\beta\gamma)_{ijk} = \sum_j (\tau\beta\gamma)_{ijk}=\sum_k (\tau\beta\gamma)_{ijk}=0$$
\end{center}
 where $y_{ijk}$ is the $ijk^{th}$ observed score, $\mu$ is the overall mean, $\tau_i$ is the effect of the $i^{th}$ temperature, $\beta_j$ is the effect of the $j^{th}$ cycle time, $\gamma_k$ is the effect of the $k^{th}$ operator, cross-terms are for interactions and $\epsilon_{ijk}$ is a random error that incorporate all sources of variability.\\
 
 
 This model is three-way ANOVA. \\
$H_0$ for temperature : $ \tau_1 = \tau_2 =0$\\
$H_1$ for cycle time : $ \beta_1 = \beta_2=\beta_3=0$\\
$H_2$ for operator : $ \gamma_1 = \gamma_2=\gamma_3=0$\\
$H_3$ for interactions : cross-terms equal 0
\begin{verbatim}
Analysis of Variance Table

Response: y
              Df Sum Sq Mean Sq F value    Pr(>F)    
ope            2 279.00 139.500 40.5000 6.111e-10 ***
temp           1  44.46  44.463 12.9086 0.0009703 ***
time           2 426.33 213.167 61.8871 2.240e-12 ***
ope:temp       2   8.04   4.019  1.1667 0.3228988    
ope:time       4 349.67  87.417 25.3790 4.765e-10 ***
temp:time      2  68.93  34.463 10.0054 0.0003504 ***
ope:temp:time  4  38.41   9.602  2.7876 0.0408701 *  
Residuals     36 124.00   3.444                          
\end{verbatim}
All effects and interactions significantly differ from 0 at the level test of significance $\alpha=0.05$, except the interaction between operator and temperature.\\


We use the Shapiro-wilk's test with the residuals to test the normality assumption, and the Levene's test for homoscedasticity.
\begin{verbatim}
        Shapiro-Wilk normality test

data:  residuals(model) 
W = 0.9709, p-value = 0.2104
\end{verbatim}
\begin{verbatim}
Levene's Test for Homogeneity of Variance (center = median)
      Df F value Pr(>F)
group 17  0.3817 0.9816
      36                            
\end{verbatim}
We can not reject the hypothesis of homoscedasticity (homogeneity) neither the hypothesis of normality.
\subparagraph{b)}
The operator could actually be considered a block variable. 
The model for the the data is \begin{center}
$$
y_{ijk}=\mu +\tau_i+\beta_j+ \gamma_k + (\tau\beta)_{ij}+ \epsilon_{ijk} \left \{
    \begin{array}{ll}
        i=1,2 \\
		j=1,2,3 \\
		k=1,2,3 
    \end{array}
\right.
\text{ with } \epsilon_{ijk} \sim \mathcal{N}(0,\sigma^2)
$$
$$\sum_i\tau_i=\sum_j\beta_i=\sum_k\gamma_k=\sum_i (\tau\beta)_{ij} = \sum_j (\tau\beta)_{ij} =0$$
 
\end{center}
 This model is three-way ANOVA. \\
$H_0$ for temperature : $ \tau_1 = \tau_2 =0$\\
$H_1$ for cycle time : $ \beta_1 = \beta_2=\beta_3=0$\\
$H_2$ for operator (blocking factor) : $ \gamma_1 = \gamma_2=\gamma_3=0$\\
$H_3$ for interactions : $(\tau\beta)_{ij}=0$ for all $i,j$
\begin{verbatim}
Analysis of Variance Table

Response: y
          Df Sum Sq Mean Sq F value    Pr(>F)    
ope        2 279.00 139.500 12.3377 5.132e-05 ***
temp       1  44.46  44.463  3.9324   0.05335 .  
time       2 426.33 213.167 18.8530 1.047e-06 ***
temp:time  2  68.93  34.463  3.0480   0.05714 .  
Residuals 46 520.11  11.307                         
\end{verbatim}
The effects of operator (block) and cycle time significantly differ from 0 at the level test of significance $\alpha=0.05$, contrasting the interaction between temperature and cycle time which is not.
\begin{verbatim}
        Shapiro-Wilk normality test

data:  residuals(mod) 
W = 0.9899, p-value = 0.9296                        
\end{verbatim}
The hypothesis of normality still holds.
\paragraph{Exercise 3.}
\subparagraph{a)}
Because we have data only for a small proportion of batches, we use a random effects model : 
\begin{center}
$$
y_{ij}=\mu +\tau_i +\epsilon_{ij} \left \{
    \begin{array}{ll}
        i=1,2,3,4,5 \\
		j=1,2,3,4,5 \\
    \end{array}
\right.
$$
\end{center}
with $\tau_{i} \sim \mathcal{N}(0,\sigma_{\tau}^2)$ and $\epsilon_{ij} \sim \mathcal{N}(0,\sigma^2)$,
 where $y_{ij}$ is the $ij^{th}$ observed calcium content, $\mu$ is the overall mean, $\tau_i$ is the random effect of the $i^{th}$ batch and $\epsilon_{ij}$ is a random error that incorporate other sources of variability.\\
We test hypotheses about the variance component $\sigma_{\tau}^2$ : \\
$H_0$ : $\sigma_{\tau}^2=0$\\
Under the null hypothesis, $F_0=\dfrac{MS_{Treatments}}{MS_E} \sim F(4,20) $
\begin{verbatim}
	Sum of Squares table
	
Source  SS      df      MS      F       p-value
Batch   0.1     4       0.02    5.98    0.002467929     
Error   0.08    20      0                              
\end{verbatim}
We reject the null hypothesis at the level test of significance $\alpha=0.05$, there is a significant variation in calcium content from batch to batch.
\subparagraph{b)}
The estimators of the components of variance are :\\
$$\widehat{\sigma}^2=MS_E=0.00417 \qquad \widehat{\sigma_{\tau}}^2=\dfrac{MS_{Treatments}-MS_E}{5}=0.0041548$$
\subparagraph{c)}
The proportion of total variance that is caused by variation between batches equal $\dfrac{\widehat{\sigma_{\tau}}^2}{\widehat{\sigma_{\tau}}^2+\widehat{\sigma^2}}=0.4990871$.
Furthermore, \\
$$\dfrac{MS_{Treatments}/(5\widehat{\sigma_{\tau}}^2+ \widehat{\sigma^2)}}{MS_E/ \widehat{\sigma}^2} \sim F(4,20)$$
Thus, the 95\% confidence interval for the proportion is : \\
$$\dfrac{L}{1+L} \leq \dfrac{\widehat{\sigma_{\tau}}^2}{\widehat{\sigma_{\tau}}^2+\widehat{\sigma^2}} \leq \dfrac{U}{1+U} \iff 
0.1234543 \leq \dfrac{\widehat{\sigma_{\tau}}^2}{\widehat{\sigma_{\tau}}^2+\widehat{\sigma^2}} \leq 0.9094268 $$
\begin{center}
Where $L=\dfrac{1}{5} \left( \dfrac{MS_{Treatments}}{MS_E*F_{0.025}(4,20)} -1 \right) $ and $U=\dfrac{1}{5} \left( \dfrac{MS_{Treatments}}{MS_E*F_{0.975}(4,20)} -1 \right)$
\end{center}
\subparagraph{d)}
We use the Shapiro-wilk test with the residuals to test the normality assumption.
\begin{verbatim}
        Shapiro-Wilk normality test

data:  residuals(m1) 
W = 0.949, p-value = 0.2381  
\end{verbatim}
We can not reject the normality hypothesis at the level test of significance $\alpha=0.05$. \\
\begin{figure}[H]
	  \centering
  	\includegraphics[width=0.6\textwidth]{4_3.png}
  	\caption{\textbf{Plotted residuals against $\widehat{y}_{ij}$.} \textsl{Roughly, we cannot reject the hypothesis of independence and homoscedasticity.}
  	}
	\end{figure}
\end{document}












