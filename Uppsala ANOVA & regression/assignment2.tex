\documentclass{article}

\usepackage[T1]{fontenc}
\usepackage[utf8]{inputenc}
%\usepackage[french]{babel}
\usepackage{graphicx}
\usepackage{lmodern}
\usepackage{amsmath}
\usepackage{amsthm}
\usepackage{listings}
\usepackage{stmaryrd}
\usepackage{enumerate}
\usepackage{amssymb}
\usepackage{amsfonts}
\usepackage{float}
\usepackage{fullpage}

\author{Latrille Thibault\\
\small thibault.latrille@ens-lyon.fr\\[-0.8ex]
\small Uppsala Universitet\\}
\title{Regression and analysis of variance \\ Home assignment 2}  
\newtheorem*{theo}{Theorem} 
\begin{document}
\maketitle
\paragraph{Exercise 1.}
The model for the the data is $$
y_{ij}=\mu +\tau_i +\epsilon_{ij} \left \{
    \begin{array}{ll}
        i=1,2,3 \\
        1j, 3j  \in \llbracket 1,8 \rrbracket^2 \\
        2j  \in \llbracket 1,12 \rrbracket
    \end{array}
\right.
$$
where $y_{ij}$ is the $ij^{th}$ observed reaction time to correct the condition, $\mu$ is the overall mean, $\tau_i$ is the effect of the $i^{th}$ arrangement of instruments, and $\epsilon_{ij}$ is a random error that incorporate all sources of variability.\\

This model is a single-factor analysis of variance (ANOVA).\\
$H_0$ : $ \tau_1 = \tau_2 = \tau_3 =0$
\begin{verbatim}
Analysis of Variance Table

Response: time
            Df Sum Sq Mean Sq F value    Pr(>F)    
arrangement  2 81.429  40.714  11.309 0.0003177 ***
Residuals   25 90.000   3.600                      
\end{verbatim}
We can reject the null hypothesis at the level test of significance $\alpha=0.01$. We conclude the arrangement of instruments does have an effect on the reaction time required to correct the emergency condition. \\ 
The assumptions for ANOVA is normality, homoscedasticity and independence of errors.\\
We use the Shapiro-wilk normality test with the residuals.
\begin{verbatim}
          Shapiro-Wilk normality test

data:  residuals(model) 
W = 0.967, p-value = 0.504
\end{verbatim}
We can not reject the normality hypothesis. 

\begin{figure}[H]
	  \centering
  	\includegraphics[width=0.6\textwidth]{4_1.png}
  	\caption{\textbf{Plotted residuals against $\widehat{y}_{ij}$.} \textsl{Roughly, we cannot reject the hypothesis of independence and homoscedasticity, which is confirmed by the Levene's test.}
  	}
	\end{figure}
	\begin{verbatim}
Levene's Test for Homogeneity of Variance (center = median)
      Df F value Pr(>F)
group  2  0.0687 0.9338
      25  
\end{verbatim}
We can not reject the hypothesis of homoscedasticity (homogeneity). Moreover, the Tukey's HSD state that arrangement 1 significantly differs from both arrangement 2 and 3.
\begin{verbatim}
  Tukey multiple comparisons of means
    95% family-wise confidence level

Fit: aov(formula = time ~ arrangement, data = ech)

$arrangement
      diff       lwr        upr     p adj
A2-A1 -2.5 -4.657122 -0.3428778 0.0208576
A3-A1 -4.5 -6.863009 -2.1369911 0.0002074
A3-A2 -2.0 -4.157122  0.1571222 0.0728959
\end{verbatim}
\begin{figure}[H]
	  \centering
  	\includegraphics[width=0.6\textwidth]{2_1_tukey.png}
  	\caption{\textbf{Tukey 95\% Simultaneous Confidence Intervals.} \textsl{Arrangement 1 significantly differs  from both arrangement 2 and 3. Arrangement 2 and 3 are not significantly different.}
  	}
	\end{figure}

\paragraph{Exercise 2.}
The model for the the data is $$
y_{ij}=\mu +\tau_i+\beta_j +\epsilon_{ij} \left \{
    \begin{array}{ll}
        i=1,2,3 \\
		j=1,2,3,4
    \end{array}
\right.
$$
where $y_{ij}$ is the $ij^{th}$ observed bacteria growth, $\mu$ is the overall mean, $\tau_i$ is the effect of the $i^{th}$ washing solution (treatment factor), $\beta_j$ is the effect of the $j^{th}$ day (blocking factor), and $\epsilon_{ij}$ is a random error that incorporate all sources of variability.\\

This model is a randomized complete block design. \\
$H_0$ for treatments : $ \tau_1 = \tau_2 = \tau_3 =0$\\
$H_0$ for blocks : $ \beta_1 = \beta_2 = \beta_3 =\beta_4=0$
\begin{verbatim}
Analysis of Variance Table

Response: obs
          Df  Sum Sq Mean Sq F value    Pr(>F)    
Solution   2  703.50  351.75  40.717 0.0003232 ***
Day        3 1106.92  368.97  42.711 0.0001925 ***
Residuals  6   51.83    8.64  
\end{verbatim}
We reject $H_0$ for both treatments and blocks at the level of significance $\alpha=0.05$. A huge proportion of variance is absorbed by the blocking factor, so days represent an important source of variability.\\
We may wonder which treatment significantly different from the others, thus we compute Tukey's HSD.
\begin{verbatim}
  Tukey multiple comparisons of means
    95% family-wise confidence level

Fit: aov(formula = values ~ solutions + day)

$solutions
      diff        lwr        upr     p adj
2-1   2.25  -4.126879   8.626879 0.5577862
3-1 -15.00 -21.376879  -8.623121 0.0008758
3-2 -17.25 -23.626879 -10.873121 0.0004067 
\end{verbatim}
The washing solution 3 significantly retards bacteria growth compared to solution 1 \& 2.
\paragraph{Exercise 3.}
The model for the the data is $$
y_{ijk}=\mu +\tau_i+\beta_j +(\tau\beta)_{ij}+\epsilon_{ijk} \left \{
    \begin{array}{ll}
        i=1,2,3 \\
		j=1,2\\
		k=1,2,3
    \end{array}
\right.
$$
where $y_{ijk}$ is the $ijk^{th}$ observed life length (in hours), $\mu$ is the overall mean, $\tau_i$ is the effect of the $i^{th}$ air temperature level, $\beta_j$ is the effect of the $j^{th}$ material for the plates, $(\tau\beta)_{ij}$ is the effect of the interaction between $\tau_i$ and $\beta_j$, and $\epsilon_{ijk}$ is still a random error that incorporate all sources of variability.\\

This model is two-way ANOVA. \\
$H_0$ for air temperature levels : $ \tau_1 = \tau_2 = \tau_3 =0$\\
$H_1$ for materials : $ \beta_1 = \beta_2=0$\\
$H_2$ for interactions : $(\tau\beta)_{ij}=0$ for all $i,j$
\begin{verbatim}
Analysis of Variance Table

Response: values
             Df  Sum Sq Mean Sq F value  Pr(>F)  
material      1  2812.5  2812.5  1.9793 0.18483  
air           2 15518.1  7759.1  5.4605 0.02059 *
material:air  2  4561.0  2280.5  1.6049 0.24118  
Residuals    12 17051.3  1420.9                  
\end{verbatim}
We can not reject the null hypothesis for the materials at the level of significance $\alpha=0.05$, neither for interactions. However, we can reject the null hypothesis for air temperature level. Moreover, there is a significant difference between the low and high air temperature, as proved by the Tukey's HSD test shown below. 
\begin{verbatim}
  Tukey multiple comparisons of means
    95% family-wise confidence level

Fit: aov(formula = values ~ material + air + material * air)

$air
                 diff       lwr       upr     p adj
low-high     71.83333  13.77139 129.89527 0.0161623
medium-high  32.83333 -25.22861  90.89527 0.3214939
medium-low  -39.00000 -97.06194  19.06194 0.2136567
\end{verbatim}
\begin{figure}[H]
	  \centering
  	\includegraphics[width=0.6\textwidth]{2_3_inter1.png}
  	\caption{\textbf{Plotted life length against air temperature level for the two materials.} \textsl{A significant interaction is indicated by a lack of parallelism in the lines, which is not the case here.}}
	\end{figure}
\end{document}
















