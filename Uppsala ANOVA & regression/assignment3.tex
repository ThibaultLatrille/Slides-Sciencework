\documentclass{article}

\usepackage[T1]{fontenc}
\usepackage[utf8]{inputenc}
%\usepackage[french]{babel}
\usepackage{graphicx}
\usepackage{lmodern}
\usepackage{amsmath}
\usepackage{amsthm}
\usepackage{listings}
\usepackage{stmaryrd}
\usepackage{enumerate}
\usepackage{amssymb}
\usepackage{cancel}
\usepackage{amsfonts}
\usepackage{float}
\usepackage{fullpage}

\author{Latrille Thibault\\
\small thibault.latrille@ens-lyon.fr\\[-0.8ex]
\small Uppsala Universitet\\}
\title{Regression and analysis of variance \\ Home assignment 3}  
\newtheorem*{theo}{Theorem} 
\begin{document}
\maketitle
\paragraph{Exercise 1.}


This model is a balanced incomplete block design. The cars are blocks and the types of gasoline additives are treatments.
The observed values are mileage performance characteristics.\\
$H_0$ : The additive does not have an effect on mileage performance characteristics.
\begin{verbatim}
Analysis of Variance Table

Response: values
          Df Sum Sq Mean Sq F value   Pr(>F)   
car        4  30.20  7.5500  7.3172 0.003973 **
add        4  31.40  7.8500  7.6079 0.003426 **
Residuals 11  11.35  1.0318                      
\end{verbatim}
We can reject the null hypothesis at the level test of significance $\alpha=0.05$. We conclude the additive does have an effect on the mileage performance characteristics. Furthermore cars represent an important source of variability.\\
We use the Shapiro-wilk test with the residuals to test the normality assumption.
\begin{verbatim}
        Shapiro-Wilk normality test

data:  residuals(model) 
W = 0.9697, p-value = 0.7491
\end{verbatim}
We can not reject the normality hypothesis. \\
We also may wonder which treatment significantly differs from the others, thus we compute Tukey's HSD.
\begin{footnotesize}
	\begin{verbatim}
 Tukey multiple comparisons of means
    95% family-wise confidence level

Fit: aov(formula = values ~ car + add, data = frame)

$add
       diff       lwr        upr     p adj
2-1 -1.7500 -4.072894  0.5728941 0.1762447
3-1 -2.1250 -4.447894  0.1978941 0.0781159
4-1 -2.8125 -5.135394 -0.4896059 0.0164044
5-1 -3.6250 -5.947894 -1.3021059 0.0027400
3-2 -0.3750 -2.697894  1.9478941 0.9831983
4-2 -1.0625 -3.385394  1.2603941 0.5946138
5-2 -1.8750 -4.197894  0.4478941 0.1352740
4-3 -0.6875 -3.010394  1.6353941 0.8684088
5-3 -1.5000 -3.822894  0.8228941 0.2901346
5-4 -0.8125 -3.135394  1.5103941 0.7875578
\end{verbatim}
\end{footnotesize}
\begin{figure}[H]
	  \centering
  	\includegraphics[width=0.6\textwidth]{3_1_tukey.png}
  	\caption{\textbf{Tukey 95\% Simultaneous Confidence Intervals.} \textsl{The additive 1 significantly differs from additive 4 \& 5 at the level test of significance $\alpha=0.05$. Moreover, the first additive is the best one.}
  	}
	\end{figure} 


\paragraph{Exercise 2.}
\subparagraph{a)}
The model for the the data is $$
y_{ijk}=\mu +\tau_i+\beta_j+ (\beta\tau)_{ij} +\epsilon_{ijk} \left \{
    \begin{array}{ll}
        i=1,2,3 \\
		j=1,2,3 \\
		k=1,2
    \end{array}
\right.
$$
where $y_{ijk}$ is the $ijk^{th}$ observed yield of the chemical process, $\mu$ is the overall mean, $\tau_i$ is the effect of the $i^{th}$ temperature, $\beta_j$ is the effect of the $j^{th}$ pressure, and $\epsilon_{ijk}$ is a random error that incorporate all sources of variability.\\

This model is two-way ANOVA. \\
$H_0$ for temperature : $ \tau_1 = \tau_2 = \tau_3 =0$\\
$H_1$ for pressure : $ \beta_1 = \beta_2=\beta_3=0$\\
$H_2$ for interactions : $(\tau\beta)_{ij}=0$ for all $i,j$
\begin{verbatim}
Analysis of Variance Table

Response: values
                     Df  Sum Sq Mean Sq F value    Pr(>F)    
Pressure              2 0.76778 0.38389 21.5937 0.0003673 ***
Temperature           2 0.30111 0.15056  8.4688 0.0085392 ** 
Pressure:Temperature  4 0.06889 0.01722  0.9688 0.4700058    
Residuals             9 0.16000 0.01778                      
\end{verbatim}
We reject $H_0$ and $H_1$ for both pressure and temperature at the level test of significance $\alpha=0.05$. The interaction is not significant. \\
\subparagraph{b)}

We use the Shapiro-wilk's test with the residuals to test the normality assumption, and the Levene's test for homoscedasticity.
\begin{verbatim}
        Shapiro-Wilk normality test

data:  residuals(model) 
W = 0.8737, p-value = 0.02046
\end{verbatim}
\begin{verbatim}
Levene's Test for Homogeneity of Variance (center = median)
      Df    F value    Pr(>F)    
group  8 3.2942e+25 < 2.2e-16 ***
       9                         
\end{verbatim}
We reject both the normality and homoscedasticity hypothesis at the level test of significance $\alpha=0.05$, it result the previous ANOVA test is irrelevant.
\begin{figure}[H]
	  \centering
  	\includegraphics[width=0.6\textwidth]{3_2_plot.png}
  	\caption{\textbf{A) Plotted residuals from first replicate (red) and second replicate (blue). B) Residuals against $\pmb{\widehat{y}_{ij}}$.} \textsl{We can detect pattern in the figures, moreover the normality and homoscedasticity assumption are rejected by statistical tests.}}
	\end{figure}
It follow we have to use the Kruskal-Wallis non-parametric test with our data.
\begin{verbatim}
        Kruskal-Wallis rank sum test

data:  values by Pressure by Temperature 
Kruskal-Wallis chi-squared = 10.4007, df = 2, p-value = 0.005515
\end{verbatim}
The non-parametric test is conclusive, we can not accept $H_0$ neither $H_1$ .
\subparagraph{c)}

The interaction is not significant, thus we can compute Tukey's HSD.
\begin{verbatim}
  Tukey multiple comparisons of means
    95% family-wise confidence level

Fit: aov(formula = values ~ Pressure + Temperature + Pressure * Temperature)

$Temperature
              diff         lwr        upr     p adj
160-150 -0.1666667 -0.38159536 0.04826203 0.1313152
170-150  0.1500000 -0.06492869 0.36492869 0.1809078
170-160  0.3166667  0.10173797 0.53159536 0.0066518

$Pressure
              diff        lwr         upr     p adj
215-200  0.3166667  0.1017380  0.53159536 0.0066518
230-200 -0.1833333 -0.3982620  0.03159536 0.0944905
230-215 -0.5000000 -0.7149287 -0.28507131 0.0002951
\end{verbatim}
However this test is not relevant because the assumption of normality and homoscedasticity is not fulfilled. 
\subparagraph{d)}
An interaction plot is more informative.
\begin{figure}[H]
	  \centering
  	\includegraphics[width=0.6\textwidth]{3_2_inter.png}
  	\caption{\textbf{Plotted yield against temperature level for the three pressure.} \textsl{A significant interaction is indicated by a lack of parallelism in the lines, which is not the case here. The higher yield of the chemical process is obtain for $P=215psig$ and $T=170^0C$}}
	\end{figure}
\newpage
\paragraph{Exercise 3.}
\subparagraph{a)}
\begin{align*}
 SS_{Treat} &= \dfrac{1}{b} \sum_{i=1}^a \left( \sum_{j=1}^b y_{ij} \right) ^2- \dfrac{1}{N}\left( \sum_{i=1}^a \sum_{j=1}^b y_{ij} \right)^2 \\ 
 &= \dfrac{1}{b} \sum_{i=1}^a \left( \sum_{j=1}^b \mu + \tau_i + \beta_j + \epsilon_{ij} \right) ^2-\dfrac{1}{N}\left( \sum_{i=1}^a \sum_{j=1}^b \mu + \tau_i +\beta_j + \epsilon_{ij} \right)^2 \\
 &= \dfrac{1}{b} \sum_{i=1}^a \left( b\mu + b\tau_i + \cancel{\sum_{j=1}^b\beta_j} + \sum_{j=1}^b \epsilon_{ij} \right) ^2- \dfrac{1}{N}\left(N\mu + a\cancel{\sum_{j=1}^b\beta_j} +b\cancel{\sum_{i=1}^a\tau_i} + \sum_{i=1}^a \sum_{j=1}^b\epsilon_{ij} \right)^2\\
 E(SS_{Treat}) &= \dfrac{1}{b} \sum_{i=1}^a \left( b^2\mu^2 + b^2\tau_i^2 + \cancel{2 \mu b^2\tau_i} + E(\sum_{j=1}^b \epsilon_{ij})^2 \right)- \dfrac{1}{N}\left(N^2\mu^2 + E(\sum_{i=1}^a \sum_{j=1}^b\epsilon_{ij})^2 \right), \\
 & \qquad \text{Since $E(\epsilon_{ij})=0$ for all $(i,j)$ and thus all cross-product involving $\epsilon_{ij}$ equal 0}. \\
&= N\mu^2+ b\sum_{i=1}^a\tau_i^2 + a \sigma^2 - N\mu^2- \sigma^2=b\sum_{i=1}^a\tau_i^2 +(a-1)\sigma^2, \\
& \qquad \text{since $E(\epsilon_{ij}\epsilon_{kl})=0$ if $(i,j)$ $\neq$  (k,l)} \\
E(MS_{Treat}) &= \dfrac{E(SS_{Treat})}{a-1} \\
 &= \sigma^2+ \dfrac{b}{a-1}\sum_{i=1}^a\tau_i^2
\end{align*}
\subparagraph{b}


$H_0$ : $\dfrac{\sum_{i=1}^a\tau_i^2}{\sigma^2}=\theta$\\


We first know that $E(SS_E)=\sigma^2$ and we get from Cochran's theorem that $\dfrac{SS_E}{\sigma^2}$ and 
$\dfrac{SS_{Treat}}{E(SS_{Treat})}$ 
are independent and are chi-squared distributed with respectively (a-1)(b-1) and a-1 degrees of freedom. \\


It follow $\dfrac{SS_E/\sigma^2}{SS_{Treat}/(\sigma^2(1+ \dfrac{\theta b}{a-1}))}=\dfrac{SS_E(1+ \dfrac{\theta b}{a-1})}{SS_{Treat}} \sim F(\mbox{\scriptsize(a-1)(b-1),a-1})$, \\
Thus we can construct the confidence interval for $\theta$ with confidence level $1-\alpha$.
\begin{center}

$$F_{\alpha/2}(\mbox{\scriptsize(a-1)(b-1),a-1}) < \dfrac{SS_E(1+ \dfrac{\theta b}{a-1})}{SS_{Treat}} <F_{1-\alpha/2}(\mbox{\scriptsize(a-1)(b-1),a-1}) $$


$$\iff \dfrac{a-1}{b}\left( \dfrac{SS_{Treat}}{SS_{E}*F_{1-\alpha/2}(\mbox{\scriptsize a-1,(a-1)(b-1)})}-1\right)  
<\theta <
\dfrac{a-1}{b}\left( \dfrac{SS_{Treat}*F_{1-\alpha/2}(\mbox{\scriptsize(a-1)(b-1),a-1})}{SS_{E}}-1\right) $$
\end{center}
\end{document}















