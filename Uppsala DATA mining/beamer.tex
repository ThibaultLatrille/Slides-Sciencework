\documentclass[10pt]{beamer}
 
\usepackage[T1]{fontenc}
\usepackage[utf8]{inputenc}
\usepackage[french]{babel}
\usepackage{graphicx}
\usepackage{hyperref}
\usepackage{lmodern}
\usepackage{listings}
\usepackage{amssymb} 
\usepackage{verbatim}
\usetheme{Warsaw}
\setbeamercovered{transparent}


\author{Susanne Bornelöv \and Thibault Latrille \and Yiming Zhang}
\title{Computer-intensive statistics and data mining}  
\institute{Uppsala Universitet, Mathematics department}
\date{14 December 2012}
 
 
\begin{document}

\frame{\titlepage} 

\section{Overview of the data}

\begin{frame}
\frametitle{What are we talking about}
	\begin{center}
       \includegraphics[width=7cm]{paper.png}
	\end{center}
\end{frame}

\begin{frame}
\frametitle{The original data}
	\begin{center}
       \includegraphics[width=9cm]{data.png}
	\end{center}
\end{frame}

\section{Kernel density estimation}
\subsection{Estimating the density}

\begin{frame}
\frametitle{Kernel density estimation}
	\begin{center}
       \includegraphics[width=8cm]{distribution.png}
	\end{center}
\end{frame}

\subsection{Bandwidth selection}

\begin{frame}
\frametitle{\textbf{Divide ut regnes,} the cross-validation method.}
	\begin{center}
       \includegraphics[width=11cm]{bandwidth.png}\\
            \begin{scriptsize}
  Let us divide our dataset $\pmb{X}$ in two set : \\
  Training test $\pmb{X}_{Training}$ and Testing set $\pmb{X}_{Test}$. \\ Then for a given bandwidth $h$, compute the kernel density estimation $\widehat{f}_h$ on $\pmb{X}_{Training}$. \\The "goodness of fit" for a particular $h$ is defined as : 
       $ FIT(h)=\sum_{X_i \in \pmb{X}_{Test}}  \widehat{f}_h(X_i)^2 $.\\
       And we minimize $FIT(h)$ regarded to $h$ with your favourite method. 
\end{scriptsize}
     \end{center}
\end{frame}

 \section{Kernel regression smoothing}

\begin{frame}
\frametitle{Kernel regression smoothing}
	\begin{center}
       \includegraphics[width=10cm]{regression0.png}
	\end{center}
\end{frame}


\begin{frame}
\frametitle{Kernel regression smoothing \& Derivative}
	\begin{center}
       \includegraphics[width=10cm]{regression.png}
	\end{center}
\end{frame}

\begin{frame}	
\frametitle{\textbf{Divide \& Conquer}, return of the cross-validation.}
	\begin{center}
       \includegraphics[width=11cm]{regressionCV.png}
       \begin{scriptsize}\\
  Again, we divide our dataset, $(\pmb{X},\pmb{Y})=(\pmb{X},\pmb{Y})_{Training} \cup (\pmb{X},\pmb{Y})_{Test}$  \\ For a given bandwidth $h$, we compute the regression estimation $\widehat{m}_h$ on $(\pmb{X},\pmb{Y})_{Training}$. \\The "Squared error" is defined as : 
       $ Error(h)=\sum_{(X_i,Y_i) \in (\pmb{X},\pmb{Y})_{Test}} \left( Y_i- \widehat{m}_h(X_i) \right) ^2 $.\\
       And we minimize $Error(h)$ regarded to $h$ with your favourite method. 
\end{scriptsize}
	\end{center}
\end{frame}


\section{Biological perspectives.}

\subsection{The interesting lifespan of the Pink Salmon}
\begin{frame}	
\frametitle{The long journey of a Bella Coola Pink Salmon}
	\begin{center}
       \includegraphics[width=11cm]{salmon.png}
	\end{center}
\end{frame}

\subsection{Odd and even years}
\begin{frame}	
\frametitle{Different population ?}
	\begin{center}
       \includegraphics[width=10cm]{greenred.png}
	\end{center}
\end{frame}

\begin{frame}	
\frametitle{When we realise our study was pointless...}
	\begin{center}
       \includegraphics[width=7cm]{merge.png}
	\end{center}
\begin{scriptsize}
\begin{center}
 $H_0$: Odd and even years salmon have the same mean weight. \\
 $p_{value} = 8.298e^{-5}$, we strongly reject $H_0$ with the non parametric Wilcoxon rank sum test.
\end{center}
\begin{center}
$H_0$: Population size is the same for even and odd years. \\
$p_{value} = 0.000522$, we strongly reject $H_0$ with the non parametric Wilcoxon rank sum test.
\end{center}
\end{scriptsize}
\end{frame}

\begin{frame}	
\frametitle{Again and again}
	\begin{center}
       \includegraphics[width=11cm]{regressionsplit.png}
	\end{center}
\end{frame}

\begin{frame}	
\frametitle{A bit of evoluationary biology}
\begin{center}
Alternative hypothesis : the decrease of both population size and mean weight is a consequence of overfishing.
\end{center}
\begin{center}
Indeed there is a high selective pressure on big salmons since small ones are more likely to swim trough netting.
\end{center} 
\begin{center}
       \includegraphics[width=9cm]{evolution.png}
	\end{center}
\end{frame}

\end{document}


