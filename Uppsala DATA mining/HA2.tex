\documentclass{article}

\usepackage[T1]{fontenc}
\usepackage[utf8]{inputenc}
%\usepackage[french]{babel}
\usepackage{graphicx}
\usepackage{lmodern}
\usepackage{amsmath}
\usepackage{amsthm}
\usepackage{listings}
\usepackage{stmaryrd}
\usepackage{enumerate}
\usepackage{amssymb}
\usepackage{dsfont}
\usepackage{cancel}
\usepackage{amsfonts}
\usepackage{float}
\usepackage{fullpage}

\author{\and Susanne Bornelöv \and Thibault Latrille \and Yiming Zhang}
\title{Computer-intensive statistics and data mining
 \\ Home assignment 2}  
\newtheorem*{theo}{Theorem} 
\begin{document}
\maketitle
For the following discussion, let $\Phi(x,\mu,\sigma)$  and $\phi(x,\mu,\sigma)$ denote respectively the cdf and the pdf of the $\mathcal{N}(\mu,\sigma^2)$ distribution.

\paragraph{Exercise 1.}
$$ \mu=\int_{-2}^0 e^{-(x-1)^2}dx=\sqrt{\pi}E_{\mathcal{N}(1,\frac{1}{2})}\mathds{1}_{[-2,0]}=\sqrt{\pi} \left[ \Phi (0,1,\frac{1}{\sqrt{2}}) - \Phi(-2,1,\frac{1}{\sqrt{2}}) \right] =0.1393832$$
Computing $\mu$ in a deterministic fashion is realized by the following R code :
\begin{verbatim}
mu<-sqrt(pi)*(pnorm(sqrt(2)*(0-1))-pnorm(sqrt(2)*(-2-1)))
\end{verbatim}
\paragraph{Exercise 2.}

$$\displaystyle \mu=\sqrt{\pi}E_{\mathcal{N}(1,\frac{1}{2})}\mathds{1}_{[-2,0]}=\int_{-\infty}^{\infty} \sqrt{\pi} \mathds{1}_{[-2,0]}(x) \Phi (x,1,1 / \sqrt{2}) dx$$ Thus by the central limit theorem, $\displaystyle \widehat{\mu}=\dfrac{1}{n}\sum_{i=1}^n \sqrt{\pi} \mathds{1}_{[-2,0]}(X_i)$, where $X_i$ are i.i.d. r.v. from the $\mathcal{N}(1,\frac{1}{2})$ distribution. Computing an approximation is straightforward as described below.
\begin{verbatim}
data<-rnorm(n,1,1/sqrt(2))
data<-data[(data>-2 & data<0)]
mu=sqrt(pi)*length(data)/n
\end{verbatim}
However, the distribution is symmetric and the integral could be rewritten as : $$\displaystyle \mu=\frac{\sqrt{\pi}}{2}E_{\mathcal{N}(1,\frac{1}{2})}\mathds{1}_{[-2,0]\cup [2,4]}=\int_{-\infty}^{\infty} \dfrac{\sqrt{\pi}}{2} \mathds{1}_{[-2,0]\cup [2,4]}(x) \Phi (x,1,1 / \sqrt{2}) dx$$
This reformulation improve the accuracy of the approximation since we use twice more data with the following algorithm
\begin{verbatim}
data<-rnorm(n,1,1/sqrt(2))
data<-data[((data>-2 & data<0))|((data>2 & data<4))]
mu=sqrt(pi)/2*length(data)/n
\end{verbatim}
For a sample of size 10000, we get $\widehat{\mu}=0.1396268$, thus the discrepancy of the independent MC is $10^{-3}$ for a sample of size 10000.

\paragraph{Exercise 3.\\}
For the hit and miss procedure we generate a sample of two dimensional independent uniform random variables such that they cover the area (see figure 1.).
For $x \in [-2,0]$ the function $f(x)=e^{-(x-1)^2}$ is increasing, thus the maximum value is achieved for $x=0$ and $e^{-1}=0.3678794$. \\
Let $A_i$ be i.i.d. r.v. from $U[-2,0]$ and $B_i$ be i.i.d. r.v. from $U[0,e^{-1}]$, thus for a sample of size n, $\displaystyle \widehat{\mu}=\dfrac{2e^{-1}}{n}\sum_{i=1}^n  \mathds{1}_{A_i < f(B_i) }$.  For a sample of size 10000, we get $\widehat{\mu}=0.143473$. 
\begin{verbatim}
f<-function(x) exp(-(x-1)**2)
a=runif(n,-2,0)
b=runif(n,0,exp(-1))
mu=(2*exp(-1)/n)*length(b[b<f(a)]) 
\end{verbatim}
\begin{figure}[H]
	  \centering
  	\includegraphics[width=0.9\textwidth]{HA2_3.png}
  	\caption{\textbf{Hit and miss procedure.} \textsl{The approximation of the integral equal the ratio between the number of dots below the curve and the total number of dots time the surface of the area.}
  	}
	\end{figure}
	
\paragraph{Exercise 4.\\}
For the importance sampling procedure, the cauchy distribution is a convenient instrumental distribution for the normal distribution as target. 
Define $\omega(x)=\dfrac{\phi(x,1,1/\sqrt{2})}{g(x)}$ where $g(x)$ is the pdf of the cauchy distribution with location parameter equal 1, then 
$\displaystyle \widehat{\mu} = \dfrac{ \sum_{i=1}^n \sqrt{\pi} \mathds{1}_{[-2,0]\cup [2,4]} (X_i) \omega(X_i) } {2\sum_{i=1}^n \omega(X_i)}$, where $X_i$ are iid from the cauchy distribution. Computing an approximation is straightforward as described below.
\begin{verbatim}
data<-rcauchy(n,1)
w<-function(x) dnorm(x,1,1/sqrt(2))/dcauchy(x,1)
wdata<-w(data[((data>-2 & data<0))|((data>2 & data<4))])
mu=sqrt(pi)/2*sum(wdata)/sum(w(data))
\end{verbatim}
For a sample of size 10000, we get $\widehat{\mu}=0.1386784$. 

\paragraph{Exercise 5.\\}
\subparagraph{a)}
The algorithm to produce the MCMC Random-walk generating $\mathcal{N}(0,\frac{1}{2})$ is : \\
Given current state $x^{(t)}$.
\begin{enumerate}
\item Draw $\epsilon \sim U(-a,a)$, where $a$ is the scaling parameter.\\
Set $y=x^{(t)}+\epsilon$
\item Draw $U \sim U[0,1]$\\
$$x^{(t+1)} = \left\{ 
\begin{array}{l l}
  y & \quad \text{if $U \leq min(1,e^{y^2-(x^{(t)})^2})$}\\
  x^{(t)} & \quad \text{else}\\ \end{array} \right.
$$
\end{enumerate}
We start with a random number (seed) and continue until we reach the sample size. The R code is given below.
\begin{verbatim}
MCMC<-function(a,seed,N){
rand<-rep(NA,N);
rand[1]<-seed;
for(i in 2:N) {
	rand[i]<-seed+a*runif(1,-1,1);
	r<-min(1,exp((seed-1)^2-(rand[i]-1)^2));
	if (runif(1)<r){seed<-rand[i]}else{rand[i]<-seed}
	};
return(rand)
}
\end{verbatim}

\subparagraph{b)}
The trial distribution is a uniform distribution, so :\\
$$T(x,y) = \left\{ 
\begin{array}{l l}
  \frac{1}{2a} & \quad \text{if $|x-y|<a$}\\
  0 & \quad \mbox{else}\\ \end{array} \right.$$
   Thus $T(x,y)=T(y,x)$ and we can compute the transition function :
$$A(x,y) = \left\{ 
\begin{array}{l l}
  \frac{min(1,e^{y^2-x^2})}{2a} & \quad \text{if $|x-y|<a$}\\
  0 & \quad \text{else}\\ \end{array} \right. 
$$
\subparagraph{c)}
In order to maximize the acceptance probability, the scaling parameter should be chosen as small as possible.
\subparagraph{d)}
A time index, $t_0>0$, defines the length of the initial non-adaptation period and we let 
$$C_t = \left\{ 
\begin{array}{l l}
  C_0 & \quad t \leq t_0 \\
  s_d Cov(X_0, \ldots, X_{t-1}) + s_d \epsilon I_D & \quad t > t_0\\ \end{array} \right. 
$$
 Recall the definition of the empirical covariance  $Cov(X_0, \ldots, X_k)=\frac{1}{k} \left( 
\sum_{i=0}^k X_i X_i- (k+1)\overline{X}_{k}\overline{X}_{k} \right) $, for $t > t_0 $, $C_t$ satisfies the recursive formula : 
$$ C_{t+1}=\frac{t-1}{t}C_t+\dfrac{s_d}{t}\left(t\overline{X}_{t-1} \overline{X}_{t-1}-(t+1)\overline{X}_{t} \overline{X}_{t}+X_t X_t + \epsilon I_d \right) $$
This permit the calculation of $C_t$ without excessive computational cost since the mean, $\overline{X}_t$, also satisfies an obvious recursive formula.\\
Following \textit{A. G. Gelman, G. O. Roberts and W. R. Gilks: Efficient Metropolis jumping rules, 1996}, we take the scaling parameter to be $s_d=(2.4)^2$  \\
For each step $(>t_0)$, using the relation $Var(U[-a,a])=\frac{a^2}{3}$, we introduce the covariance into the Random-walk setting $a=\sqrt{3C_t}$. 
\subparagraph{d)}
The R function that produce the adaptive MCMC is :
\begin{verbatim}
AMCMC<-function(seed,N,to){
N<-N+to
rand<-rep(NA,N);
C0<-1
Sd<-2.4^2
eps<-0.1
rand[1]<-seed;
for(i in 2:to) {
	rand[i]<-seed+sqrt(3*C0)*runif(1,-1,1);
	r<-min(1,exp((seed-1)^2-(rand[i]-1)^2));
	if (runif(1)<r){seed<-rand[i]}else{rand[i]<-seed}
 }
Mean<-mean(rand[1:to])
C0<-(sum(rand[1:to])-to*Mean**2)/(to-1)
for (i in (to+1):N) {
	t=i-1
	rand[i]<-seed+sqrt(3*abs(C0))*runif(1,-1,1);
	r<-min(1,exp((seed-1)^2-(rand[i]-1)^2));
	if (runif(1)<r){seed<-rand[i]} else{rand[i]<-seed};
	NMean<-(Mean*(i-1)+rand[i])/i
	C0<-(t-1)*C0/t+Sd*(t*Mean**2-i*NMean**2+rand[i]**2+eps)/t
	Mean<-NMean
     }
return(rand[(to+1):N])
}
\end{verbatim}
\subparagraph{e)}
By the ergodic property of the markov chain, $\displaystyle \widehat{\mu}=\dfrac{1}{2n}\sum_{i=1}^n \sqrt{\pi} \mathds{1}_{[-2,0]\cup [2,4] }(X_i)$, where $X_i$ are the random variables generated by the adaptive MCMC algorithm. Computing the integral is straightforward regarding to preceding exercise. The R code is given below.
\begin{verbatim}
data<-AMCMC(0,n,(n/10))
length(data)
data<-data[((data>-2 & data<0))|((data>2 & data<4))]
mu=sqrt(pi)/2*length(data)/n
\end{verbatim}
For a sample of size 10000, we get $\widehat{\mu}=0.1378969$.
\end{document}












